\section{Token List and String}

A token list is a sequence of tokens, which can be used to store data in \LaTeX{}.
It is very easy to create a token list: simply enclose the data with braces \inlpl|{...}|, and a token list is formed.
A string in \LTT{} is a special type of token list that contains only string tokens.
In this section, we show examples of creating, manipulating, and using token lists and strings.

In \cref{ex:tl-ops}, token lists are used as a container to store and access data.
In the example, we build a simple student management system that allows one to obtain student names by their IDs.
We also define a function that outputs all students stored.
\begin{latexsample}[exampletitle={Token List Operations},examplelabel={ex:tl-ops}]
  % declare functions in LaTeX3 mode
  \ExplSyntaxOn
  % create a token list to store students
  \tl_new:N \l_student_db_tl
  % create a temporary token list
  \tl_new:N \l_student_tmp_tl
  \newcommand{\AddStudent}[1]
  {
    % append data to the token list
    \tl_put_right:Nn \l_student_db_tl {{#1}}
  }
  \newcommand{\GetStudentName}[1]
  {
    % get data by index
    \tl_item:Nn \l_student_db_tl {#1}
  }
  \newcommand{\GetAllStudentName}
  {
    % clear temporary token list
    \tl_clear:N \l_tmp_student_tl
    % iterate over each item in the token list
    % the content of each item will be stored in ##1 in loop body
    \tl_map_inline:Nn \l_student_db_tl
    {
      \tl_put_right:Nn \l_tmp_student_tl {##1,}
    }
    % use the temporary token list in the document
    \tl_use:N \l_tmp_student_tl
  }
  \ExplSyntaxOff

  % use functions in normal mode
  \AddStudent{Alice}
  \AddStudent{Bob}
  \par\GetStudentName{1}
  \par\GetStudentName{2}
  \par\GetAllStudentName
  \AddStudent{Mary}
  \AddStudent{John}
  \par\GetAllStudentName
\end{latexsample}