\usepackage[T1]{fontenc}
\usepackage{lmodern}
\usepackage{amsmath, amssymb}

\DeclareFontFamily{T1}{DejaVuSansMono-TLF}{\hyphenchar\font-1}
\DeclareFontShape{T1}{DejaVuSansMono-TLF}{m}{n}{<-> s*[0.75]DejaVuSansMono-tlf-t1}{}
\DeclareFontShape{T1}{DejaVuSansMono-TLF}{m}{it}{<-> s*[0.75]DejaVuSansMono-Oblique-tlf-t1}{}
\DeclareFontShape{T1}{DejaVuSansMono-TLF}{b}{n}{<-> s*[0.75]DejaVuSansMono-Bold-tlf-t1}{}
\DeclareFontShape{T1}{DejaVuSansMono-TLF}{b}{it}{<-> s*[0.75]DejaVuSansMono-BoldOblique-tlf-t1}{}

\DeclareFontFamily{T1}{lmtt}{\hyphenchar\font-1}
\DeclareFontShape{T1}{lmtt}{m}{n}{<-> s*[0.9]ec-lmtt10}{}
\DeclareFontShape{T1}{lmtt}{m}{it}{<-> s*[0.9]ec-lmtti10}{}
\DeclareFontShape{T1}{lmtt}{m}{sl}{<-> s*[0.9]ec-lmtto10}{}
\usepackage[english]{babel}
\usepackage{metalogo}
\usepackage{tcolorbox}
\usepackage{booktabs}
\usepackage{microtype}
\usepackage{expl3}
\usepackage{mdframed}
\usepackage{chngcntr}
\usepackage{etoolbox}
\usepackage{adjustbox}
\usepackage{supertabular}
\usepackage{makecell}
\usepackage{tikz}
\usepackage{ragged2e}
\usepackage{float}
\usepackage{array}
\usepackage{xcolor}
\usepackage{datetime2}
\usepackage{listings}
\usepackage{xsimverb}
\usepackage[scale=0.9]{tgheros}

\ExplSyntaxOn

% overleaf mode is considered debug mode
\bool_new:N \g_doc_overleaf_mode_bool
\bool_gset_false:N \g_doc_overleaf_mode_bool
\cs_set:Npn \UseOverleafMode
{
    \bool_gset_true:N \g_doc_overleaf_mode_bool
}
\ExplSyntaxOff


%\usepackage[breaklinks,colorlinks,pdfa]{hyperref}
\usepackage[breaklinks]{hyperref}
\usepackage[capitalise]{cleveref}


\tcbuselibrary{listings, minted, skins, breakable, xparse, hooks}

% a quick command for LaTeX3
\newcommand*{\LT}{\texorpdfstring{\LaTeX}{LaTeX}}
\newcommand*{\LTT}{\texorpdfstring{\LaTeX3}{LaTeX3}}
\let\liii\LTT


% set the style of the list of examples
% https://tex.stackexchange.com/questions/86711/tcolorbox-list-of-listings
\newcommand{\ListOfCodeExampleName}{List of Examples}
\makeatletter
\newcommand{\ListOfCodeExample}{\section*{\ListOfCodeExampleName}\@starttoc{CodeExample}}
\makeatother

% the counter for numbered code examples
\newcounter{codeexample}
\counterwithin{codeexample}{section}

%%%%%%%%%%%%%%%%%%%%%%%%%%%%%%%%%%%%%%%%%%%%%%%%%%%%%%%%%%%%%%%%%%
% this segment of code is to implement arbitrary code
% cross-referencing within the article
%%%%%%%%%%%%%%%%%%%%%%%%%%%%%%%%%%%%%%%%%%%%%%%%%%%%%%%%%%%%%%%%%%
\makeatletter
\ExplSyntaxOn
\bool_new:N \g_lst_line_ref_bool
\tl_new:N \g_lst_label_tl
\tl_new:N \l_lst_tmpa_tl
\tl_new:N \l_lst_tmpb_tl

\newcommand{\SetLstRefLabel}[1]{\tl_gset:Nn \g_lst_label_tl {#1}}


\cs_set:Npn \__lst_render_line: {
    \textcolor[rgb]{0.5,0.5,0.5}{
    \ttfamily\scriptsize
    \arabic{FancyVerbLine}
    }
}

\cs_set:Npn \__lst_make_hypertarget:nn #1 #2{
   \raisebox{1em}{\hypertarget{#1}{}}#2 
}

\renewcommand{\theFancyVerbLine}{
    \tl_if_empty:NTF \g_lst_label_tl {
        \__lst_render_line:
    }{
        \tl_set:Nx \l_lst_tmpa_tl {\arabic{FancyVerbLine}}
        \exp_args:Nx \__lst_make_hypertarget:nn {\g_lst_label_tl-\l_lst_tmpa_tl}{
            \__lst_render_line:
        }
    }
}


\seq_new:N \l_lst_line_seg_seq
\tl_new:N \l_lst_first_line_tl
\DeclareDocumentCommand{\lref}{mm}{
    \regex_split:nnN {-} {#2} \l_lst_line_seg_seq
    \int_case:nnF {\seq_count:N \l_lst_line_seg_seq}
    {
        {1} {Line~}
        {2} {Lines~}
    }
    {
        \GenericError{}{\string\lref~received~invalid~arguments}{}{}
    }

    \seq_pop_left:NN \l_lst_line_seg_seq \l_lst_first_line_tl

    \exp_args:Nx \hyperlink{#1-\l_lst_first_line_tl}
    {
        #2
    }
    @\ref{#1}
}



\bool_new:N \ExportCurrentCodeExample
\bool_new:N \CurrentExampleHasTitle
\let\SetBoolTrue\bool_gset_true:N
\let\SetBoolFalse\bool_gset_false:N
\let\BoolShow\bool_show:N
\def\CurrentExampleRawTitle{}
\ExplSyntaxOff
\makeatother
%%%%%%%%%%%%%%%%%%%%%%%%%%%%%%%%%%%%%%%%%%%%%%%%%%%%%%%%%%%%%%%%%%
%%%%%%%%%%%%%%%%%%%%%%%%%%%%%%%%%%%%%%%%%%%%%%%%%%%%%%%%%%%%%%%%%%

% basic style for examples
% options for controlling which listing engine to use
\ExplSyntaxOn
\bool_new:N \g_doc_use_minted_bool
\bool_gset_false:N \g_doc_use_minted_bool
\cs_new:Npn \UseMintedHighlighting
{
    \bool_gset_true:N \g_doc_use_minted_bool
}
\cs_new:Npn \MintedHighlightingTF #1#2
{
    \bool_if:NTF \g_doc_use_minted_bool 
    {#1} {#2}
}
\ExplSyntaxOff

\tcbset{
    commonstyle/.style={
        enhanced,
        breakable,
        beforeafter skip=3ex,
        fontlower=\sffamily\small,
        fonttitle=\normalfont,
        colback=white,
        colframe=black!60,
        boxrule=0.8pt,
        left=1mm,
        right=1mm,
        top=0.5mm,
        bottom=0.5mm,
    }
}


\tcbset{
    noexec/.style={listing only},
    examplelabel/.code args={#1}{
        \SetLstRefLabel{#1}
    },
    exampletitle/.style args={#1}{
        title={Example \thecodeexample: #1},
        after app={\SetBoolTrue\CurrentExampleHasTitle\gdef\CurrentExampleRawTitle{#1}}
    },
    noexport/.code={
        \SetBoolFalse\ExportCurrentCodeExample
    }
}

\makeatletter
\ExplSyntaxOn

% options for controlling example export globally
\bool_new:N \g_lst_export_example_bool
\bool_gset_false:N \g_lst_export_example_bool
\cs_new:Npn \ExportExamples
{
    \bool_gset_true:N \g_lst_export_example_bool
}


\tl_new:N \l_doc_tmpa_tl
\tl_new:N \ExampleFilename
\cs_set:Npn \UpdateExampleFilename {
    \tl_set_eq:NN \l_doc_tmpa_tl \thecodeexample
    \regex_replace_all:nnN {\.} {-} \l_doc_tmpa_tl
    \tl_set_eq:NN \ExampleFilename \l_doc_tmpa_tl
}

\iow_new:N \g_lst_example_iow
\ior_new:N \g_lst_example_ior
\tl_new:N \l_lst_now_time_tl
\def\CurrentExampleTitle{}

\cs_set:Npn \doc_include_latex_example:n #1 {

    \refstepcounter{codeexample}

    \SetBoolTrue\ExportCurrentCodeExample
    \SetBoolFalse\CurrentExampleHasTitle
    \SetLstRefLabel{}

    \tl_gset:Nx \CurrentExampleTitle {Example~\thecodeexample}


    % load the listing file
    \exp_args:Nx \tcbinputlisting {
        codesample,
        listing~file={\jobname-latexsample.vrb},
        title={\exp_not:V \CurrentExampleTitle},
        #1
    }

    % add to contents
    \bool_if:NT \CurrentExampleHasTitle
    {
        \addcontentsline{CodeExample}{subsection}{\protect\numberline{\thecodeexample}\CurrentExampleRawTitle}
    }

    % add labels
    \tl_if_empty:NF \g_lst_label_tl
    {   
        \exp_args:NV \label \g_lst_label_tl
    }

    % export example
    \bool_if:nT {\g_lst_export_example_bool && \ExportCurrentCodeExample}
    {
        \UpdateExampleFilename
        \iow_open:Nn \g_lst_example_iow {examples/example-\ExampleFilename.tex}
        \ior_open:Nn \g_lst_example_ior {temp-example.vrb}

    
        \iow_now:Nx \g_lst_example_iow {\c_percent_str\space \CurrentExampleTitle}
        \iow_now:Nx \g_lst_example_iow {\c_percent_str\space This~file~is~exported~by~\jobname.tex}
        \iow_now:Nx \g_lst_example_iow {\c_percent_str\space   Do~not~modify~this~file~directly;~the~changes~will~not~be~reflected~in~the~source~document}

        \tl_set:Nx \l_lst_now_time_tl {\DTMNow}
        \regex_replace_all:nnN {\c{relax}} {} \l_lst_now_time_tl
        \iow_now:Nx \g_lst_example_iow {\c_percent_str\space Export~timestamp:~\l_lst_now_time_tl}
        
        \ior_str_map_inline:Nn \g_lst_example_ior
        {
            \iow_now:Nn \g_lst_example_iow {##1}
        }
        \ior_close:N \g_lst_example_ior
        \iow_close:N \g_lst_example_iow
    }
}

\DeclareDocumentEnvironment{latexsample}{O{}}{
    \XSIMfilewritestart{\jobname-latexsample.vrb}
}
{
    \XSIMfilewritestop
    \group_begin:
    \doc_include_latex_example:n {#1}
    \group_end:
}
\ExplSyntaxOff
\makeatother

\ExplSyntaxOn
\newcommand*{\GenAnonTitle}[1]{
    \refstepcounter{codeexample}
    \tl_if_empty:nTF {#1}
    {
        Example~\thecodeexample
    }
    {
        Example~\thecodeexample: 
    }
}
\ExplSyntaxOff


\makeatletter
% better URL line breaking
\g@addto@macro{\UrlBreaks}{\UrlOrds}
\makeatother

\crefname{codeexample}{Example}{Examples}

\AtBeginDocument
{
    % define listing related macros
    \MintedHighlightingTF
    {
        \newmintinline[inltex]{tex_lexer.py:Tex3Lexer}{breaklines, breakanywhere, fontfamily=DejaVuSansMono-TLF,style=colorful}
        \newmintinline[inlpy]{python}{fontfamily=DejaVuSansMono-TLF,style=colorful}
        \newmintinline[inlpl]{text}{}

        \tcbset{
            codesample/.style={
                commonstyle,
                listing engine=minted,
                minted language=tex_lexer.py:Tex3Lexer,
                minted options={
                fontsize=\fontsize{9}{9},
                autogobble,
                breaklines,
                breakanywhere,
                obeytabs,
                tabsize=2,
                linenos,
                numbersep=2mm,
                xleftmargin=4mm,
                fontfamily=DejaVuSansMono-TLF,
                style=colorful
                }
            }
        }

        % listing of any language
        \newtcblisting{codesample}[1]{
            codesample,
            listing only,
            minted language=#1
        }
    }
    {
        \definecolor{mygreen}{rgb}{0,0.6,0}
        \definecolor{mygray}{rgb}{0.5,0.5,0.5}
        \definecolor{mymauve}{rgb}{0.58,0,0.82}

        \lstset{ 
            basicstyle=\small\ttfamily,        % the size of the fonts that are used for the code
            breakatwhitespace=false,         % sets if automatic breaks should only happen at whitespace
            breaklines=true,                 % sets automatic line breaking
            commentstyle=\color{mygreen},    % comment style
            %deletekeywords={...},            % if you want to delete keywords from the given language
            %escapeinside={\%*}{*)},          % if you want to add LaTeX within your code
            %extendedchars=true,              % lets you use non-ASCII characters; for 8-bits encodings only, does not work with UTF-8
            %firstnumber=1000,                % start line enumeration with line 1000
            %frame=single,	                   % adds a frame around the code
            keepspaces=true,                 % keeps spaces in text, useful for keeping indentation of code (possibly needs columns=flexible)
            keywordstyle=\color{blue},       % keyword style
            %language=Octave,                 % the language of the code
            %morekeywords={*,...},            % if you want to add more keywords to the set
            %numbers=left,                    % where to put the line-numbers; possible values are (none, left, right)
            %numbersep=5pt,                   % how far the line-numbers are from the code
            %numberstyle=\tiny\color{mygray}, % the style that is used for the line-numbers
            %stepnumber=2,                    % the step between two line-numbers. If it's 1, each line will be numbered
            %stringstyle=\color{mymauve},     % string literal style
            tabsize=4,	                   % sets default tabsize to 2 spaces
            %title=\lstname                   % show the filename of files included with \lstinputlisting; also try caption instead of title
        }

        \let\inltex\lstinline
        \let\inply\lstinline
        \let\inlpl\lstinline

        \tcbset{
            codesample/.style={
                commonstyle,
                left=1em,
                listing engine=listings,
                listing options={
                    basicstyle=\scriptsize\ttfamily,        % the size of the fonts that are used for the code
                    breakatwhitespace=false,         % sets if automatic breaks should only happen at whitespace
                    breaklines=true,                 % sets automatic line breaking
                    commentstyle=\color{mygreen},    % comment style
                    %deletekeywords={...},            % if you want to delete keywords from the given language
                    %escapeinside={\%*}{*)},          % if you want to add LaTeX within your code
                    %extendedchars=true,              % lets you use non-ASCII characters; for 8-bits encodings only, does not work with UTF-8
                    %firstnumber=1000,                % start line enumeration with line 1000
                    %frame=single,	                   % adds a frame around the code
                    keepspaces=true,                 % keeps spaces in text, useful for keeping indentation of code (possibly needs columns=flexible)
                    keywordstyle=\color{blue},       % keyword style
                    language=tex,                 % the language of the code
                    %morekeywords={*,...},            % if you want to add more keywords to the set
                    numbers=left,                    % where to put the line-numbers; possible values are (none, left, right)
                    numbersep=5pt,                   % how far the line-numbers are from the code
                    numberstyle=\scriptsize\ttfamily\color{mygray}, % the style that is used for the line-numbers
                    %stepnumber=2,                    % the step between two line-numbers. If it's 1, each line will be numbered
                    %stringstyle=\color{mymauve},     % string literal style
                    tabsize=4,	                   % sets default tabsize to 2 spaces
                    %title=\lstname                   % show the filename of files included with \lstinputlisting; also try caption instead of title
                }
            }
        }
    }
}