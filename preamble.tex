\usepackage[T1]{fontenc}
\usepackage{lmodern}
\usepackage{amsmath, amssymb}

\DeclareFontFamily{T1}{DejaVuSansMono-TLF}{\hyphenchar\font-1}
\DeclareFontShape{T1}{DejaVuSansMono-TLF}{m}{n}{<-> s*[0.75]DejaVuSansMono-tlf-t1}{}
\DeclareFontShape{T1}{DejaVuSansMono-TLF}{m}{it}{<-> s*[0.75]DejaVuSansMono-Oblique-tlf-t1}{}
\DeclareFontShape{T1}{DejaVuSansMono-TLF}{b}{n}{<-> s*[0.75]DejaVuSansMono-Bold-tlf-t1}{}
\DeclareFontShape{T1}{DejaVuSansMono-TLF}{b}{it}{<-> s*[0.75]DejaVuSansMono-BoldOblique-tlf-t1}{}

\DeclareFontFamily{T1}{lmtt}{\hyphenchar\font-1}
\DeclareFontShape{T1}{lmtt}{m}{n}{<-> s*[0.9]ec-lmtt10}{}
\DeclareFontShape{T1}{lmtt}{m}{it}{<-> s*[0.9]ec-lmtti10}{}
\DeclareFontShape{T1}{lmtt}{m}{sl}{<-> s*[0.9]ec-lmtto10}{}
\usepackage[english]{babel}
\usepackage{metalogo}
\usepackage{tcolorbox}
\usepackage{booktabs}
\usepackage{microtype}
\usepackage{expl3}
\usepackage{mdframed}
\usepackage{chngcntr}
\usepackage{etoolbox}
\usepackage{adjustbox}
\usepackage{supertabular}
\usepackage{makecell}
\usepackage{tikz}
\usepackage{ragged2e}
\usepackage{float}
\usepackage{array}
\usepackage{xcolor}
%\usepackage[breaklinks,colorlinks,pdfa]{hyperref}
\usepackage[breaklinks]{hyperref}
\usepackage[capitalise]{cleveref}


\tcbuselibrary{listings, minted, skins, breakable, xparse, hooks}

% a quick command for LaTeX3
\newcommand*{\liii}{\LaTeX3}


% using custom lexer for highlighting LaTeX3 syntax
\newcommand*{\liiilexer}{tex_lexer.py:Tex3Lexer -x}
% other code highliting setup
\newmintinline[inltex]{\liiilexer}{breaklines, breakanywhere, fontfamily=DejaVuSansMono-TLF}
\newmintinline[inlpy]{python}{fontfamily=DejaVuSansMono-TLF}
\newmintinline[inlpl]{text}{}

% set the style of the list of examples
% https://tex.stackexchange.com/questions/86711/tcolorbox-list-of-listings
\newcommand{\ListOfCodeExampleName}{List of Examples}
\makeatletter
\newcommand{\ListOfCodeExample}{\section*{\ListOfCodeExampleName}\@starttoc{CodeExample}}
\makeatother

% the counter for numbered code examples
\newcounter{codeexample}
\counterwithin{codeexample}{section}

%%%%%%%%%%%%%%%%%%%%%%%%%%%%%%%%%%%%%%%%%%%%%%%%%%%%%%%%%%%%%%%%%%
% this segment of code is to implement arbitrary code
% cross-referencing within the article
%%%%%%%%%%%%%%%%%%%%%%%%%%%%%%%%%%%%%%%%%%%%%%%%%%%%%%%%%%%%%%%%%%
\makeatletter
\ExplSyntaxOn
\bool_new:N \g_lst_line_ref_bool
\tl_new:N \g_lst_label_tl
\tl_new:N \l_lst_tmpa_tl
\tl_new:N \l_lst_tmpb_tl
\prop_new:N \g_lst_index_prop

\cs_set:Npn \__lst_write_aux:n #1 {
    \immediate\write\@auxout{#1}
}

\newcommand{\EnableLstRef}{\bool_gset_true:N \g_lst_line_ref_bool}
\newcommand{\DisableLstRef}{\bool_gset_false:N \g_lst_line_ref_bool}
\newcommand{\SetLstRefLabel}[1]{\tl_gset:Nn \g_lst_label_tl {#1}}
\newcommand{\LstPutLookup}[1]{
    \tl_set:Nn \l_lst_tmpb_tl {
       \string\expandafter\string\gdef\string\csname\space @@lst-lookup-#1\string\endcsname{\TheLstCounter}
    }
    \exp_args:Nx \__lst_write_aux:n \l_lst_tmpb_tl
}
\newcommand{\LstUseLookup}[1]{
    \cs_if_exist:cTF {@@lst-lookup-#1} {
        \use:c {@@lst-lookup-#1}
    } {
        ?
    }
}


\cs_set:Npn \__lst_render_line: {
    \textcolor[rgb]{0.5,0.5,0.5}{
    \ttfamily\scriptsize
    \arabic{FancyVerbLine}
    }
}

\cs_set:Npn \__lst_make_hypertarget:nn #1 #2{
   \raisebox{1em}{\hypertarget{#1}{}}#2 
}

\renewcommand{\theFancyVerbLine}{
    \bool_if:NTF \g_lst_line_ref_bool {
        \tl_set:Nx \l_lst_tmpa_tl {\arabic{FancyVerbLine}}
        \exp_args:Nx \__lst_make_hypertarget:nn {\g_lst_label_tl-\l_lst_tmpa_tl}{
            \__lst_render_line:
        }
    }{
        \__lst_render_line:
    }

}

\int_new:N \g_lst_counter_int
\newcommand{\AddLstCounter}{\int_gincr:N \g_lst_counter_int}
\newcommand{\TheLstCounter}{\int_use:N \g_lst_counter_int}

% user command to cross reference a line in the lising
% #1: listing name
% #2: line number
% #3 (optional): would output a range of lines from #2 to #3 if present
\DeclareDocumentCommand{\lref}{mmo}{
    \IfValueTF{#3}{Lines}{Line}~\hyperlink{#1-#2}{
        \LstUseLookup{#1}:
        \IfValueTF{#3}{#2-#3}{#2}
    }
}

% user command to cross referenec a listing
\DeclareDocumentCommand{\lstref}{m}{
    Listing~\hyperlink{#1-1}{
        \LstUseLookup{#1}
    }
}

\ExplSyntaxOff
\makeatother
%%%%%%%%%%%%%%%%%%%%%%%%%%%%%%%%%%%%%%%%%%%%%%%%%%%%%%%%%%%%%%%%%%
%%%%%%%%%%%%%%%%%%%%%%%%%%%%%%%%%%%%%%%%%%%%%%%%%%%%%%%%%%%%%%%%%%

% basic style for examples
\tcbset{
  codesample/.style={
    enhanced,
    breakable,
    beforeafter skip=3ex,
    listing engine=minted,
    minted language=\liiilexer,
    fontlower=\sffamily\small,
    minted options={
      fontsize=\fontsize{9}{9},
      autogobble,
      breaklines,
      obeytabs,
      tabsize=2,
      linenos,
      numbersep=2mm,
      xleftmargin=4mm,
      fontfamily=DejaVuSansMono-TLF
    },
    fonttitle=\normalfont,
    colback=white,
    colframe=black!60,
    boxrule=0.8pt,
    left=1mm,
    right=1mm,
    top=0.5mm,
    bottom=0.5mm,
    before upper pre={\AddLstCounter},
    overlay unbroken and last={
        \node[font=\sffamily\bfseries\scriptsize, anchor=south east] at (frame.south east) {\color[rgb]{0.5,0.5,0.5} \TheLstCounter};
    }
  }
}

%%%%%%%%%%%%%%%%%%%%%%%%%%%%%%%%%%%%%%%%%%%%%%%%%%%%%%%%%%%%%%%%%%
% this segment of code is to implement code exporting
% if \ExportExamplestrue is called, then the source code of each
% example will be output to /examples
%%%%%%%%%%%%%%%%%%%%%%%%%%%%%%%%%%%%%%%%%%%%%%%%%%%%%%%%%%%%%%%%%%
\newif\ifExportExamples

\ifExportExamples

% define our custom VerbatimOut like environment for latexsample
\makeatletter

\ExplSyntaxOn

\cs_set:Npn \__doc_write:Nn #1#2 {
    \immediate\write#1{#2}
}
\cs_generate_variant:Nn \__doc_write:Nn {Nx}
\cs_set_eq:NN \MyWriteA \__doc_write:Nn
\cs_set_eq:NN \MyWriteB \__doc_write:Nx
\cs_set_eq:NN \PCTSGN \c_percent_str

\cs_set:Npn \__doc_open_out:Nn #1#2 {
    \immediate\openout#1 #2\relax
}
\cs_generate_variant:Nn \__doc_open_out:Nn {Nx}
\cs_set_eq:NN \MyOpenOut \__doc_open_out:Nx

\tl_new:N \l_doc_tmpa_tl
\tl_new:N \ExampleFilename

\cs_set:Npn \UpdateExampleFilename {
    \tl_set_eq:NN \l_doc_tmpa_tl \thecodeexample
    \regex_replace_all:nnN {\.} {-} \l_doc_tmpa_tl
    \tl_set_eq:NN \ExampleFilename \l_doc_tmpa_tl
}

\ExplSyntaxOff

\newwrite\FV@OutFileA
\newwrite\FV@OutFileB

\def\MyProcessLine#1{%
    \immediate\write\FV@OutFileA{#1}%
    \immediate\write\FV@OutFileB{#1}%
}

\def\latexsample{\FV@Environment{}{latexsample}}
\def\FVB@latexsample#1{%
  \@bsphack
  \begingroup
    \FV@UseKeyValues
    \FV@DefineWhiteSpace
    \def\FV@Space{\space}%
    \FV@DefineTabOut
    \let\FV@ProcessLine\MyProcessLine
    % increment example counter
    \refstepcounter{codeexample}
    \UpdateExampleFilename
    %\def\FV@ProcessLine{\immediate\write\FV@OutFile}%
    % copy1: this is to be read back and compiled in the document
    \MyOpenOut\FV@OutFileA{temp-example.vrb}
    %\immediate\openout\FV@OutFileA temp-example.vrb\relax
    % copy2: this is to be saved in examples folder
    %\immediate\openout\FV@OutFileB temp-example2.vrb\relax
    \MyOpenOut\FV@OutFileB{examples/example-\ExampleFilename.tex}
    % write the comment in the example
    \MyWriteB\FV@OutFileB{\PCTSGN\PCTSGN\space Example \thecodeexample: #1}
    % save #1
    \gdef\TempExampleTitle{#1}
    \let\FV@FontScanPrep\relax
%% DG/SR modification begin - May. 18, 1998 (to avoid problems with ligatures)
    \let\@noligs\relax
%% DG/SR modification end
    \FV@Scan}
\def\FVE@latexsample{%
\immediate\closeout\FV@OutFileA%
\immediate\closeout\FV@OutFileB%
\endgroup\@esphack%
% input listing
\inputlatexsample{\TempExampleTitle}{temp-example.vrb}
}

\DefineVerbatimEnvironment{latexsample}{latexsample}{}

\makeatother


\newtcbinputlisting{\inputlatexsample}[2]{
  codesample,
  title=\GenExampleTitle{#1},
  listing file={#2}
}

\newcommand{\GenExampleTitle}[1]{%
    %\refstepcounter{codeexample}
    \hspace*{0.5em}
    Example \thecodeexample:~#1
    \addcontentsline{CodeExample}{subsection}{\protect\numberline{\thecodeexample}#1}
}

%TODO: fix the environment when exporting!

\else

\newcommand{\GenExampleTitle}[2]{%
    \refstepcounter{codeexample}
    \IfValueT{#2}{\label{#2}}
    \hspace*{0.5em}
    Example \thecodeexample:~#1
    \addcontentsline{CodeExample}{subsection}{\protect\numberline{\thecodeexample}#1}
}

\DeclareTCBListing{latexsample}{m!o}{
    codesample,
    title=\GenExampleTitle{#1}{#2},
    IfValueTF={#2}
    {
        before upper app={\EnableLstRef\SetLstRefLabel{#2}\LstPutLookup{#2}},
        after app={\DisableLstRef}
    }
    {}
}

%\newtcblisting{latexsample}[1]{
%  codesample,
%  title=\GenExampleTitle{#1}
%}

\fi


%%%%%%%%%%%%%%%%%%%%%%%%%%%%%%%%%%%%%%%%%%%%%%%%%%%%%%%%%%%%%%%%%%
%%%%%%%%%%%%%%%%%%%%%%%%%%%%%%%%%%%%%%%%%%%%%%%%%%%%%%%%%%%%%%%%%%



% listing only example
\DeclareTCBListing{latexsample**}{!o}{
    codesample,
    listing only,
    IfValueTF={#1}
    {
        before upper app={\EnableLstRef\SetLstRefLabel{#1}\LstPutLookup{#1}},
        after app={\DisableLstRef}
    }
    {}
}

% anonymous example
\DeclareTCBListing{latexsample*}{!o}{
    codesample,
    IfValueTF={#1}
    {
        before upper app={\EnableLstRef\SetLstRefLabel{#1}\LstPutLookup{#1}},
        after app={\DisableLstRef}
    }
    {}
}

%\newtcblisting{latexsample*}{
%  codesample
%}

% listing of any language
\newtcblisting{codesample}[1]{
  codesample,
  listing only,
  minted language=#1
}



\makeatletter
% better URL line breaking
\g@addto@macro{\UrlBreaks}{\UrlOrds}

%\newcommand{\inlcolorbox}[2]{
%    \bgroup
%    %\setlength{\fboxsep}{0pt}
%    \colorbox{blue!30}{#2}
%    \egroup
%}

%\patchcmd{\minted@inputpyg@inline}{\colorbox}{\inlcolorbox}{}{\GenericError{}{cannot patch minted}{}{}}
%\let\old@minted@inputpyg@inline\minted@inputpyg@inline
%\renewcommand{cmd}{def}

\crefname{codeexample}{Example}{Examples}
\newcommand{\exfullref}[1]{Example \ref{#1} (\lstref{#1})}

\makeatother