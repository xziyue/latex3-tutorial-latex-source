%%%%%%%%%%%%%%%%%%%%%%%%%%%%%%%%%%%%PROTECT%%%%%%%%%%%%%%%%%%%%%%%%%%%%%%%%%%%%%
%%%%%%%%%%%%%%%%%%%%%%%%%%%%%%%%%%%%PROTECT%%%%%%%%%%%%%%%%%%%%%%%%%%%%%%%%%%%%%
\documentclass{ltugboat}
%%%%%%%%%%%%%%%%%%%%%%%%%%%%%%%%%%%%PROTECT%%%%%%%%%%%%%%%%%%%%%%%%%%%%%%%%%%%%%
\usepackage[T1]{fontenc}
\usepackage{lmodern}
\usepackage{amsmath, amssymb}

\DeclareFontFamily{T1}{DejaVuSansMono-TLF}{\hyphenchar\font-1}
\DeclareFontShape{T1}{DejaVuSansMono-TLF}{m}{n}{<-> s*[0.75]DejaVuSansMono-tlf-t1}{}
\DeclareFontShape{T1}{DejaVuSansMono-TLF}{m}{it}{<-> s*[0.75]DejaVuSansMono-Oblique-tlf-t1}{}
\DeclareFontShape{T1}{DejaVuSansMono-TLF}{b}{n}{<-> s*[0.75]DejaVuSansMono-Bold-tlf-t1}{}
\DeclareFontShape{T1}{DejaVuSansMono-TLF}{b}{it}{<-> s*[0.75]DejaVuSansMono-BoldOblique-tlf-t1}{}

\DeclareFontFamily{T1}{lmtt}{\hyphenchar\font-1}
\DeclareFontShape{T1}{lmtt}{m}{n}{<-> s*[0.9]ec-lmtt10}{}
\DeclareFontShape{T1}{lmtt}{m}{it}{<-> s*[0.9]ec-lmtti10}{}
\DeclareFontShape{T1}{lmtt}{m}{sl}{<-> s*[0.9]ec-lmtto10}{}
\usepackage[english]{babel}
\usepackage{metalogo}
\usepackage{tcolorbox}
\usepackage{booktabs}
\usepackage{microtype}
\usepackage{expl3}
\usepackage{mdframed}
\usepackage{chngcntr}
\usepackage{etoolbox}
\usepackage{adjustbox}
\usepackage{supertabular}
\usepackage{makecell}
\usepackage{tikz}
\usepackage{ragged2e}
\usepackage{float}
\usepackage{array}
\usepackage{xcolor}
\usepackage{datetime2}
\usepackage[scale=0.9]{tgheros}

\ExplSyntaxOn
\file_if_exist:nTF{USE_SHELL_ESCAPE}
{
    \file_if_exist:nTF{MINTED_FINALIZE}
    {
        \iow_term:n {=====~Document~is~compiled~in~finalization~mode~=====}
        \usepackage[finalizecache,cachedir={minted-frozen}]{minted}
    }
    {
        \iow_term:n {=====~Document~is~compiled~in~development~mode~=====}
        \usepackage{minted}
    }
}
{
    \iow_term:n {=====~Document~is~compiled~in~submission~mode~=====}
    \usepackage[frozencache,cachedir={minted-frozen}]{minted}
}
\ExplSyntaxOff


%\usepackage[breaklinks,colorlinks,pdfa]{hyperref}
\usepackage[breaklinks]{hyperref}
\usepackage[capitalise]{cleveref}


\tcbuselibrary{listings, minted, skins, breakable, xparse, hooks}

% a quick command for LaTeX3
\newcommand*{\LT}{\texorpdfstring{\LaTeX}{LaTeX}}
\newcommand*{\LTT}{\texorpdfstring{\LaTeX3}{LaTeX3}}
\let\liii\LTT

\renewcommand{\MintedPygmentize}{./my_pygmentize}
% other code highliting setup
\newmintinline[inltex]{tex_lexer.py:Tex3Lexer}{breaklines, breakanywhere, fontfamily=DejaVuSansMono-TLF,style=colorful}
\newmintinline[inlpy]{python}{fontfamily=DejaVuSansMono-TLF,style=colorful}
\newmintinline[inlpl]{text}{}

% set the style of the list of examples
% https://tex.stackexchange.com/questions/86711/tcolorbox-list-of-listings
\newcommand{\ListOfCodeExampleName}{List of Examples}
\makeatletter
\newcommand{\ListOfCodeExample}{\section*{\ListOfCodeExampleName}\@starttoc{CodeExample}}
\makeatother

% the counter for numbered code examples
\newcounter{codeexample}
\counterwithin{codeexample}{section}

%%%%%%%%%%%%%%%%%%%%%%%%%%%%%%%%%%%%%%%%%%%%%%%%%%%%%%%%%%%%%%%%%%
% this segment of code is to implement arbitrary code
% cross-referencing within the article
%%%%%%%%%%%%%%%%%%%%%%%%%%%%%%%%%%%%%%%%%%%%%%%%%%%%%%%%%%%%%%%%%%
\makeatletter
\ExplSyntaxOn
\bool_new:N \g_lst_line_ref_bool
\tl_new:N \g_lst_label_tl
\tl_new:N \l_lst_tmpa_tl
\tl_new:N \l_lst_tmpb_tl

\newcommand{\SetLstRefLabel}[1]{\tl_gset:Nn \g_lst_label_tl {#1}}


\cs_set:Npn \__lst_render_line: {
    \textcolor[rgb]{0.5,0.5,0.5}{
    \ttfamily\scriptsize
    \arabic{FancyVerbLine}
    }
}

\cs_set:Npn \__lst_make_hypertarget:nn #1 #2{
   \raisebox{1em}{\hypertarget{#1}{}}#2 
}

\renewcommand{\theFancyVerbLine}{
    \tl_if_empty:NTF \g_lst_label_tl {
        \__lst_render_line:
    }{
        \tl_set:Nx \l_lst_tmpa_tl {\arabic{FancyVerbLine}}
        \exp_args:Nx \__lst_make_hypertarget:nn {\g_lst_label_tl-\l_lst_tmpa_tl}{
            \__lst_render_line:
        }
    }
}

% user command to cross reference a line in the listing
% #1: listing name
% #2: line number
% #3 (optional): would output a range of lines from #2 to #3 if present
% \DeclareDocumentCommand{\lref}{mmo}{
%     \IfValueTF{#3}{Lines}{Line}~\hyperlink{#1-#2}{
%         \IfValueTF{#3}{#2-#3}{#2}
%         @
%         \LstUseLookup{#1}
%     }
% }

\seq_new:N \l_lst_line_seg_seq
\tl_new:N \l_lst_first_line_tl
\DeclareDocumentCommand{\lref}{mm}{
    \regex_split:nnN {-} {#2} \l_lst_line_seg_seq
    \int_case:nnF {\seq_count:N \l_lst_line_seg_seq}
    {
        {1} {Line~}
        {2} {Lines~}
    }
    {
        \GenericError{}{\string\lref~received~invalid~arguments}{}{}
    }

    \seq_pop_left:NN \l_lst_line_seg_seq \l_lst_first_line_tl

    \exp_args:Nx \hyperlink{#1-\l_lst_first_line_tl}
    {
        #2
    }
    @\ref{#1}
}



\bool_new:N \ExportCurrentCodeExample
\let\SetBoolTrue\bool_gset_true:N
\let\SetBoolFalse\bool_gset_false:N
\ExplSyntaxOff
\makeatother
%%%%%%%%%%%%%%%%%%%%%%%%%%%%%%%%%%%%%%%%%%%%%%%%%%%%%%%%%%%%%%%%%%
%%%%%%%%%%%%%%%%%%%%%%%%%%%%%%%%%%%%%%%%%%%%%%%%%%%%%%%%%%%%%%%%%%

% basic style for examples
\gdef\CurrentExampleTitle{}
\tcbset{
  codesample/.style={
    enhanced,
    breakable,
    beforeafter skip=3ex,
    listing engine=minted,
    minted language=tex_lexer.py:Tex3Lexer,
    fontlower=\sffamily\small,
    minted options={
      fontsize=\fontsize{9}{9},
      autogobble,
      breaklines,
      breakanywhere,
      obeytabs,
      tabsize=2,
      linenos,
      numbersep=2mm,
      xleftmargin=4mm,
      fontfamily=DejaVuSansMono-TLF,
      style=colorful
    },
    fonttitle=\normalfont,
    colback=white,
    colframe=black!60,
    boxrule=0.8pt,
    left=1mm,
    right=1mm,
    top=0.5mm,
    bottom=0.5mm,
    before upper app={\SetBoolTrue\ExportCurrentCodeExample\SetLstRefLabel{}},
  }
}


\tcbset{
    noexec/.style={listing only},
    examplelabel/.style args={#1}{
        before upper app={\SetLstRefLabel{#1}},
    },
    exampletitle/.style args={#1}{
        title={Example \thecodeexample: #1}
    },
    noexport/.code={
        \SetBoolFalse\ExportCurrentCodeExample
    }
}



\makeatletter
\ExplSyntaxOn

% options for controlling example export globally
\bool_new:N \g_lst_export_example_bool
\bool_gset_false:N \g_lst_export_example_bool
\cs_new:Npn \ExportExamples
{
    \bool_gset_true:N \g_lst_export_example_bool
}

\cs_set:Npn \__doc_write:Nn #1#2 {
    \immediate\write#1{#2}
}


\cs_generate_variant:Nn \__doc_write:Nn {Nx}
\cs_set_eq:NN \MyWriteA \__doc_write:Nn
\cs_set_eq:NN \MyWriteB \__doc_write:Nx
\cs_set_eq:NN \PCTSGN \c_percent_str

\cs_set:Npn \__doc_open_out:Nn #1#2 {
    \immediate\openout#1 #2\relax
}
\cs_generate_variant:Nn \__doc_open_out:Nn {Nx}
\cs_set_eq:NN \MyOpenOut \__doc_open_out:Nx

\tl_new:N \l_doc_tmpa_tl
\tl_new:N \ExampleFilename
\cs_set:Npn \UpdateExampleFilename {
    \tl_set_eq:NN \l_doc_tmpa_tl \thecodeexample
    \regex_replace_all:nnN {\.} {-} \l_doc_tmpa_tl
    \tl_set_eq:NN \ExampleFilename \l_doc_tmpa_tl
}

\iow_new:N \g_lst_example_iow
\ior_new:N \g_lst_example_ior
\tl_new:N \l_lst_now_time_tl

\cs_set:Npn \@@DoLaTeXExample #1 {

    \refstepcounter{codeexample}

    \tl_gset:Nx \CurrentExampleTitle {Example~\thecodeexample}

    % load the listing file
    \exp_args:Nx \tcbinputlisting { 
        codesample,
        listing~file={temp-example.vrb},
        title={\exp_not:V \CurrentExampleTitle}
        \exp_not:V #1
    }

    % add labels
    \tl_if_empty:NF \g_lst_label_tl
    {   
        \iow_term:x {=== \g_lst_label_tl : \thecodeexample}
        \exp_args:NV \label \g_lst_label_tl
    }

    % export example
    \bool_if:nT {\g_lst_export_example_bool && \ExportCurrentCodeExample}
    {
        \UpdateExampleFilename
        \iow_open:Nn \g_lst_example_iow {examples/example-\ExampleFilename.tex}
        \ior_open:Nn \g_lst_example_ior {temp-example.vrb}

        \tl_if_empty:NTF \CurrentExampleTitle
        {
            \iow_now:Nx \g_lst_example_iow {\c_percent_str\space  This~is~Example~\thecodeexample\space ~exported~by~\jobname.tex}
        }
        {
            \iow_now:Nx \g_lst_example_iow {\c_percent_str\space   This~is~exported~Example~\thecodeexample\space ~(\CurrentExampleTitle)~exported~by~\jobname.tex}
        }
        \iow_now:Nx \g_lst_example_iow {\c_percent_str\space   Do~not~modify~this~file~directly;~the~changes~will~not~be~reflected~in~the~source~document}

        \tl_set:Nx \l_lst_now_time_tl {\DTMNow}
        \regex_replace_all:nnN {\c{relax}} {} \l_lst_now_time_tl
        \iow_now:Nx \g_lst_example_iow {\c_percent_str\space Export~timestamp:~\l_lst_now_time_tl}
        
        \ior_str_map_inline:Nn \g_lst_example_ior
        {
            \iow_now:Nn \g_lst_example_iow {##1}
        }
        \ior_close:N \g_lst_example_ior
        \iow_close:N \g_lst_example_iow
    }

    \gdef\CurrentExampleTitle{}
}

\ExplSyntaxOff


% change fancyvrb environment so that we can use our own key-val options
\def\@@CurLaTeXExArgs{}
\def\FV@EnvironmentNew#1#2{%
  \def\FV@KeyValues{#1}%
  \def\@@CurLaTeXExArgs{}
  \catcode`\^^M=\active
  \@ifnextchar[%
    {\catcode`\^^M=5 \FV@GetKeyValues{\let\@@CurLaTeXExArgs\FV@KeyValues\def\FV@KeyValues{#1}\@nameuse{FVB@#2}}}%
    {\catcode`\^^M=5 \@nameuse{FVB@#2}}}

\def\FVB@latexsample{%
  \@bsphack
  \begingroup
    \FV@UseKeyValues
    \FV@DefineWhiteSpace
    \def\FV@Space{\space}%
    \FV@DefineTabOut
    \def\FV@ProcessLine{\immediate\write\FV@OutFile}%
    \immediate\openout\FV@OutFile temp-example.vrb\relax
    \let\FV@FontScanPrep\relax
%% DG/SR modification begin - May. 18, 1998 (to avoid problems with ligatures)
    \let\@noligs\relax
%% DG/SR modification end
    \FV@Scan}
\def\FVE@latexsample{\immediate\closeout\FV@OutFile\endgroup\@esphack\@@DoLaTeXExample\@@CurLaTeXExArgs}


\@namedef{latexsample}{\FV@EnvironmentNew{}{latexsample}}%
\@namedef{endlatexsample}{\@nameuse{FVE@latexsample}}%


\makeatother


% \newtcbinputlisting{\inputlatexsample}[2]{
%   codesample,
%   title=\GenExampleTitle{#1},
%   listing file={#2}
% }

% \newcommand{\GenExampleTitle}[1]{%
%     %\refstepcounter{codeexample}
%     \hspace*{0.5em}
%     Example \thecodeexample:~#1
%     \addcontentsline{CodeExample}{subsection}{\protect\numberline{\thecodeexample}#1}
% }

% %TODO: fix the environment when exporting!

% \else

% \newcommand{\GenExampleTitle}[2]{%
%     \refstepcounter{codeexample}
%     \IfValueT{#2}{\label{#2}}
%     \hspace*{0.5em}
%     Example \thecodeexample:~#1
%     \addcontentsline{CodeExample}{subsection}{\protect\numberline{\thecodeexample}#1}
% }

% \DeclareTCBListing{latexsample}{m!o}{
%     codesample,
%     title=\GenExampleTitle{#1}{#2},
%     IfValueTF={#2}
%     {
%         before upper app={\EnableLstRef\SetLstRefLabel{#2}\LstPutLookup{#2}},
%         after app={\DisableLstRef}
%     }
%     {}
% }

%\newtcblisting{latexsample}[1]{
%  codesample,
%  title=\GenExampleTitle{#1}
%}



%%%%%%%%%%%%%%%%%%%%%%%%%%%%%%%%%%%%%%%%%%%%%%%%%%%%%%%%%%%%%%%%%%
%%%%%%%%%%%%%%%%%%%%%%%%%%%%%%%%%%%%%%%%%%%%%%%%%%%%%%%%%%%%%%%%%%

\ExplSyntaxOn
\newcommand*{\GenAnonTitle}[1]{
    \refstepcounter{codeexample}
    \tl_if_empty:nTF {#1}
    {
        Example~\thecodeexample
    }
    {
        Example~\thecodeexample: 
    }
}

\ExplSyntaxOff


%\newtcblisting{latexsample*}{
%  codesample
%}

% listing of any language
\newtcblisting{codesample}[1]{
  codesample,
  listing only,
  minted language=#1
}



\makeatletter
% better URL line breaking
\g@addto@macro{\UrlBreaks}{\UrlOrds}

%\newcommand{\inlcolorbox}[2]{
%    \bgroup
%    %\setlength{\fboxsep}{0pt}
%    \colorbox{blue!30}{#2}
%    \egroup
%}

%\patchcmd{\minted@inputpyg@inline}{\colorbox}{\inlcolorbox}{}{\GenericError{}{cannot patch minted}{}{}}
%\let\old@minted@inputpyg@inline\minted@inputpyg@inline
%\renewcommand{cmd}{def}

\crefname{codeexample}{Example}{Examples}

\makeatother
%%%%%%%%%%%%%%%%%%%%%%%%%%%%%%%%%%%%PROTECT%%%%%%%%%%%%%%%%%%%%%%%%%%%%%%%%%%%%%
%%%%%%%%%%%%%%%%%%%%%%%%%%%%%%%%%%%%PROTECT%%%%%%%%%%%%%%%%%%%%%%%%%%%%%%%%%%%%%
\title{Modern \LaTeX~programming: an example based \liii~tutorial}
\author{Ziyue Xiang}
\address{Purdue University}
\netaddress{ziyue.alan.xiang (at) gmail (dot) com}
\personalURL{https://www.alanshawn.com}
\ORCID{0000-0001-6054-5801}
%%%%%%%%%%%%%%%%%%%%%%%%%%%%%%%%%%%%PROTECT%%%%%%%%%%%%%%%%%%%%%%%%%%%%%%%%%%%%%
%%%%%%%%%%%%%%%%%%%%%%%%%%%%%%%%%%%%PROTECT%%%%%%%%%%%%%%%%%%%%%%%%%%%%%%%%%%%%%
\begin{document}
%%%%%%%%%%%%%%%%%%%%%%%%%%%%%%%%%%%%PROTECT%%%%%%%%%%%%%%%%%%%%%%%%%%%%%%%%%%%%%
\begin{abstract}
The programming side of \LaTeX\ is often overlooked.
In many cases, the use of programming capabilities of \LaTeX\ can facilitate document editing and improve the quality of the compiled document.
\LaTeX3 is a set of modern programming interfaces provided by the \LaTeX\ kernel.
Compared to traditional \LaTeX\ programming, \LaTeX3 yields more standardized, readable, and robust code.
In this tutorial, we present a series of examples that help readers understand the commonly used modules of \LaTeX3.
\end{abstract}
%%%%%%%%%%%%%%%%%%%%%%%%%%%%%%%%%%%%PROTECT%%%%%%%%%%%%%%%%%%%%%%%%%%%%%%%%%%%%%
\maketitle
%%%%%%%%%%%%%%%%%%%%%%%%%%%%%%%%%%%%PROTECT%%%%%%%%%%%%%%%%%%%%%%%%%%%%%%%%%%%%%
\tableofcontents
\ListOfCodeExample
%%%%%%%%%%%%%%%%%%%%%%%%%%%%%%%%%%%%PROTECT%%%%%%%%%%%%%%%%%%%%%%%%%%%%%%%%%%%%%
%%%%%%%%%%%%%%%%%%%%%%%%%%%%%%%%%%%%PROTECT%%%%%%%%%%%%%%%%%%%%%%%%%%%%%%%%%%%%%
\section{Introduction}
%%%%%%%%%%%%%%%%%%%%%%%%%%%%%%%%%%%%PROTECT%%%%%%%%%%%%%%%%%%%%%%%%%%%%%%%%%%%%%
There is no doubt that \LaTeX~is viewed as a typesetting language by most of the users.
The programming aspect of \LaTeX~is often overlooked by many.
In practice, many large and structured documents can benefit from the programming capabilities provided by \LaTeX. 
Even understanding the most basic programming principles in \LaTeX~can greatly facilitate the efficiency of generating figures or tables that are made up of similar and repeated sub-structures.
The infrastructure provided by \LaTeXe\ \cite{berry2017latex} is already Turing complete, which means its programming capabilities are identical to those of Python \cite{vanrossum2010python} and C \cite{ritchie1988c}.
However, the syntax and conventions of \LaTeXe~are nonstandardized and obsolete compared to the mainstream programming languages now.
This makes learning \LaTeXe~programming more difficult for today's \LaTeX~users.
%%%%%%%%%%%%%%%%%%%%%%%%%%%%%%%%%%%%PROTECT%%%%%%%%%%%%%%%%%%%%%%%%%%%%%%%%%%%%%
%%%%%%%%%%%%%%%%%%%%%%%%%%%%%%%%%%%%PROTECT%%%%%%%%%%%%%%%%%%%%%%%%%%%%%%%%%%%%%
In order to modernize the programming interfaces in \LaTeX, the \liii~programming interfaces are introduced \cite{mittelbach2020quo}.
Unfortunately, the learning materials for this tool chain is scarce.
One of the few resources available for new learners is \emph{The \liii\ Interfaces} \cite{l3interface}, which is a technical documentation that is difficult to for one to start with.
Therefore, in this material we intend to provide an example based \liii\ tutorial for new \liii\ users with sufficient background in computer programming.
That is, the reader is expected to understand basic structures (e.g., loops, conditional branches) as well as data types (e.g., integers, floating point numbers, strings) in computer programs. 
%%%%%%%%%%%%%%%%%%%%%%%%%%%%%%%%%%%%PROTECT%%%%%%%%%%%%%%%%%%%%%%%%%%%%%%%%%%%%%
%%%%%%%%%%%%%%%%%%%%%%%%%%%%%%%%%%%%PROTECT%%%%%%%%%%%%%%%%%%%%%%%%%%%%%%%%%%%%%
%%%%%%%%%%%%%%%%%%%%%%%%%%%%%%%%%%%%PROTECT%%%%%%%%%%%%%%%%%%%%%%%%%%%%%%%%%%%%%
Since the \liii~project has accumulated a huge code base over the years, it is infeasible to cover all of its functionalities in one tutorial.
In this tutorial, we focus on the most frequently used components in \liii. 
The complete documentation of \liii\  can be found in \cite{l3interface}. 
In the upcoming section titles, if parentheses are present, then the content in the parentheses is the corresponding section number in the 2022-01-12 release of \cite{l3interface}.
%%%%%%%%%%%%%%%%%%%%%%%%%%%%%%%%%%%%PROTECT%%%%%%%%%%%%%%%%%%%%%%%%%%%%%%%%%%%%%
%%%%%%%%%%%%%%%%%%%%%%%%%%%%%%%%%%%%PROTECT%%%%%%%%%%%%%%%%%%%%%%%%%%%%%%%%%%%%%
%%%%%%%%%%%%%%%%%%%%%%%%%%%%%%%%%%%%PROTECT%%%%%%%%%%%%%%%%%%%%%%%%%%%%%%%%%%%%%
\subsection{Motivations of \liii}
%%%%%%%%%%%%%%%%%%%%%%%%%%%%%%%%%%%%PROTECT%%%%%%%%%%%%%%%%%%%%%%%%%%%%%%%%%%%%%
As mentioned above, \LaTeXe~is already Turing complete and serves as the building blocks of many existing packages. In this section, we describe the problems of traditional \LaTeX~programming, which justifies the reason why \liii~is developed despite already having the powerful \LaTeXe~and many other existing packages.
%%%%%%%%%%%%%%%%%%%%%%%%%%%%%%%%%%%%PROTECT%%%%%%%%%%%%%%%%%%%%%%%%%%%%%%%%%%%%%
\paragraph{Nonuniform interface} 
%%%%%%%%%%%%%%%%%%%%%%%%%%%%%%%%%%%%PROTECT%%%%%%%%%%%%%%%%%%%%%%%%%%%%%%%%%%%%%
Outside \liii, the interfaces provided by traditional \LaTeX~are not standardized. They suffer from the following disadvantages.
%%%%%%%%%%%%%%%%%%%%%%%%%%%%%%%%%%%%PROTECT%%%%%%%%%%%%%%%%%%%%%%%%%%%%%%%%%%%%%
%%%%%%%%%%%%%%%%%%%%%%%%%%%%%%%%%%%%PROTECT%%%%%%%%%%%%%%%%%%%%%%%%%%%%%%%%%%%%%
Firstly, the mechanism of \LaTeX~can affect the readability of traditional \LaTeX~code. 
In \LaTeX, functions and variables are all declared using control sequences (see Chapter 3 of \cite{knuth1984texbook}).
When a function is invoked, it is expected to execute a series of predefined procedures. Variables are used to store values only.
In \LaTeX, we can declare functions that absorb one or multiple arguments.
Sometimes arguments are stored in variables.
Since both functions and its arguments can be both control sequences, it is difficult to distinguish between them. 
This is likely to make reading traditional \LaTeX\ source code difficult.
%%%%%%%%%%%%%%%%%%%%%%%%%%%%%%%%%%%%PROTECT%%%%%%%%%%%%%%%%%%%%%%%%%%%%%%%%%%%%%
%%%%%%%%%%%%%%%%%%%%%%%%%%%%%%%%%%%%PROTECT%%%%%%%%%%%%%%%%%%%%%%%%%%%%%%%%%%%%%
Secondly, in traditional \LaTeX~the implementation of many fundamental programming capabilities are provided by external packages.
For example, to compare the equality of two strings, we can use \inlinestylea{\PY{k}{\PYZbs{}ifthenelse}} and \inlinestylea{\PY{k}{\PYZbs{}equal}} from \verb|ifthen| package \cite{pkg:ifthen};
we can use \inlinestylea{\PY{k}{\PYZbs{}pdfstrcmp}} from \verb|pdftexcmds| package \cite{pkg:pdftexcmds};
we can also use \inlinestylea{\PY{k}{\PYZbs{}IfStrEq}} from \verb|xstring| package \cite{pkg:xstring}. 
The use of multiple similar packages is likely to cause redundancy and compatibility issues.
The lack of a centralized documentation and comparison for these similar packages also increases the learning cost of traditional \LaTeX\ programming.
%%%%%%%%%%%%%%%%%%%%%%%%%%%%%%%%%%%%PROTECT%%%%%%%%%%%%%%%%%%%%%%%%%%%%%%%%%%%%%
\paragraph{Expansion control} 
%%%%%%%%%%%%%%%%%%%%%%%%%%%%%%%%%%%%PROTECT%%%%%%%%%%%%%%%%%%%%%%%%%%%%%%%%%%%%%
%%%%%%%%%%%%%%%%%%%%%%%%%%%%%%%%%%%%PROTECT%%%%%%%%%%%%%%%%%%%%%%%%%%%%%%%%%%%%%
Unlike generic programming languages, \LaTeX\ does not have support for types.
Programming components such as variables, functions and function arguments are all treated in the same way.
Therefore, \LaTeX~programmers need to more precisely control how variables are defined and how functions are called. 
These techniques are known as expansion control. 
Expansion control in traditional \LaTeX~is achieved by using the \inlinestylea{\PY{k}{\PYZbs{}expandafter}} command, which is difficult to use because the number of \inlinestylea{\PY{k}{\PYZbs{}expandafter}} command calls required may scale exponentially \cite{bechtolsheim1988tutorial}.
%%%%%%%%%%%%%%%%%%%%%%%%%%%%%%%%%%%%PROTECT%%%%%%%%%%%%%%%%%%%%%%%%%%%%%%%%%%%%%
\liii~aims at mitigating these inconveniences in traditional \LaTeX\ programming.
It provides a uniform interface for \LaTeX~programmers, where functions and variables are separated from each other. 
It defines standardized infrastructure for many programming tasks such as string processing, numerical calculation, regular expression matching, etc.
The new expansion control mechanism of \liii\ is easier and more straightforward to use.
%%%%%%%%%%%%%%%%%%%%%%%%%%%%%%%%%%%%PROTECT%%%%%%%%%%%%%%%%%%%%%%%%%%%%%%%%%%%%%
%%%%%%%%%%%%%%%%%%%%%%%%%%%%%%%%%%%%PROTECT%%%%%%%%%%%%%%%%%%%%%%%%%%%%%%%%%%%%%
%%%%%%%%%%%%%%%%%%%%%%%%%%%%%%%%%%%%PROTECT%%%%%%%%%%%%%%%%%%%%%%%%%%%%%%%%%%%%%
\subsection{Compiling Examples}
%%%%%%%%%%%%%%%%%%%%%%%%%%%%%%%%%%%%PROTECT%%%%%%%%%%%%%%%%%%%%%%%%%%%%%%%%%%%%%
This tutorial is based on examples.
To compile the examples, the minimum preamble required is:
\def\CurrentEnvName{latexsample**}
\begin{filecontents}[force,noheader]{\jobname-lst-raw.vrb}
\documentclass{article}
\usepackage{tikz} % load TikZ for some TikZ examples
\usepackage{expl3} % load latex3 packages
\end{filecontents}
\begin{filecontents}[force,noheader]{\jobname-lst-hl.vrb}
\begin{Verbatim}[commandchars=\\\{\}]
\PY{k}{\PYZbs{}documentclass}\PY{n+nb}{\PYZob{}}article\PY{n+nb}{\PYZcb{}}
\PY{k}{\PYZbs{}usepackage}\PY{n+nb}{\PYZob{}}tikz\PY{n+nb}{\PYZcb{}} \PY{c}{\PYZpc{} load TikZ for some TikZ examples}
\PY{k}{\PYZbs{}usepackage}\PY{n+nb}{\PYZob{}}expl3\PY{n+nb}{\PYZcb{}} \PY{c}{\PYZpc{} load latex3 packages}
\end{Verbatim}
\end{filecontents}
\noindent The example code should be placed between \inlinestylea{\PY{k}{\PYZbs{}begin}\PY{n+nb}{\PYZob{}}document\PY{n+nb}{\PYZcb{}}} 
and \inlinestylea{\PY{k}{\PYZbs{}end}\PY{n+nb}{\PYZob{}}document\PY{n+nb}{\PYZcb{}}}. 
All examples were tested with \TeX Live  2020 on Ubuntu 20.04.
For newer versions of \LaTeX~compilers, there is no need to load the \verb|expl3| package explicitly (i.e., \lref{ex:min-preamble}{3}\footnotemark~is optional).
%%%%%%%%%%%%%%%%%%%%%%%%%%%%%%%%%%%%PROTECT%%%%%%%%%%%%%%%%%%%%%%%%%%%%%%%%%%%%%
\footnotetext{Every listing has a unique index, which is shown at the bottom right corner. A line in the listing is referenced by <listing index>:<line number>. A range of lines in the listing are referenced by <listing index>:<line number 1>-<line number 2>.}
%%%%%%%%%%%%%%%%%%%%%%%%%%%%%%%%%%%%PROTECT%%%%%%%%%%%%%%%%%%%%%%%%%%%%%%%%%%%%%
The source code of this tutorial can be obtained from \url{https://github.com/xziyue/latex3-tutorial-latex-source}.
%%%%%%%%%%%%%%%%%%%%%%%%%%%%%%%%%%%%PROTECT%%%%%%%%%%%%%%%%%%%%%%%%%%%%%%%%%%%%%
%%%%%%%%%%%%%%%%%%%%%%%%%%%%%%%%%%%%PROTECT%%%%%%%%%%%%%%%%%%%%%%%%%%%%%%%%%%%%%
\section{\liii\ Naming Conventions (I.1.1)}
%%%%%%%%%%%%%%%%%%%%%%%%%%%%%%%%%%%%PROTECT%%%%%%%%%%%%%%%%%%%%%%%%%%%%%%%%%%%%%
Unlike many programming languages that enclose the function arguments with parentheses, \LaTeX\ does not require delimiters between the function and its arguments. 
In the example below, we define 6 control sequences, where \inlinestylea{\PY{k}{\PYZbs{}ta}} and \inlinestylea{\PY{k}{\PYZbs{}td}} are functions, and the rest are variables.
\def\CurrentEnvName{latexsample*}
\begin{filecontents}[force,noheader]{\jobname-lst-raw.vrb}
\newcommand{\ta}[2]{[arg1={#1}, arg2={#2}]}
\newcommand{\tb}{$\alpha$}
\newcommand{\tc}{$\beta$}
\newcommand{\td}[1]{[arg3={#1}]}
\newcommand{\te}{$\gamma$}
\newcommand{\tf}{$\delta$}
\ta\tb\tc\td\te\tf
\end{filecontents}
\begin{filecontents}[force,noheader]{\jobname-lst-hl.vrb}
\begin{Verbatim}[commandchars=\\\{\}]
\PY{k}{\PYZbs{}newcommand}\PY{n+nb}{\PYZob{}}\PY{k}{\PYZbs{}ta}\PY{n+nb}{\PYZcb{}}[2]\PY{n+nb}{\PYZob{}}[arg1=\PY{n+nb}{\PYZob{}}\PYZsh{}1\PY{n+nb}{\PYZcb{}}, arg2=\PY{n+nb}{\PYZob{}}\PYZsh{}2\PY{n+nb}{\PYZcb{}}]\PY{n+nb}{\PYZcb{}}
\PY{k}{\PYZbs{}newcommand}\PY{n+nb}{\PYZob{}}\PY{k}{\PYZbs{}tb}\PY{n+nb}{\PYZcb{}}\PY{n+nb}{\PYZob{}}\PY{l+s}{\PYZdl{}}\PY{n+nv}{\PYZbs{}alpha}\PY{l+s}{\PYZdl{}}\PY{n+nb}{\PYZcb{}}
\PY{k}{\PYZbs{}newcommand}\PY{n+nb}{\PYZob{}}\PY{k}{\PYZbs{}tc}\PY{n+nb}{\PYZcb{}}\PY{n+nb}{\PYZob{}}\PY{l+s}{\PYZdl{}}\PY{n+nv}{\PYZbs{}beta}\PY{l+s}{\PYZdl{}}\PY{n+nb}{\PYZcb{}}
\PY{k}{\PYZbs{}newcommand}\PY{n+nb}{\PYZob{}}\PY{k}{\PYZbs{}td}\PY{n+nb}{\PYZcb{}}[1]\PY{n+nb}{\PYZob{}}[arg3=\PY{n+nb}{\PYZob{}}\PYZsh{}1\PY{n+nb}{\PYZcb{}}]\PY{n+nb}{\PYZcb{}}
\PY{k}{\PYZbs{}newcommand}\PY{n+nb}{\PYZob{}}\PY{k}{\PYZbs{}te}\PY{n+nb}{\PYZcb{}}\PY{n+nb}{\PYZob{}}\PY{l+s}{\PYZdl{}}\PY{n+nv}{\PYZbs{}gamma}\PY{l+s}{\PYZdl{}}\PY{n+nb}{\PYZcb{}}
\PY{k}{\PYZbs{}newcommand}\PY{n+nb}{\PYZob{}}\PY{k}{\PYZbs{}tf}\PY{n+nb}{\PYZcb{}}\PY{n+nb}{\PYZob{}}\PY{l+s}{\PYZdl{}}\PY{n+nv}{\PYZbs{}delta}\PY{l+s}{\PYZdl{}}\PY{n+nb}{\PYZcb{}}
\PY{k}{\PYZbs{}ta}\PY{k}{\PYZbs{}tb}\PY{k}{\PYZbs{}tc}\PY{k}{\PYZbs{}td}\PY{k}{\PYZbs{}te}\PY{k}{\PYZbs{}tf}
\end{Verbatim}
\end{filecontents}
\noindent On \lref{ex:func-var-def}{7}, we call functions \inlinestylea{\PY{k}{\PYZbs{}ta}} and \inlinestylea{\PY{k}{\PYZbs{}td}} with their respective arguments (stored in variables). 
We output the value of \inlinestylea{\PY{k}{\PYZbs{}tf}} next.
In appearance, \lref{ex:func-var-def}{7} is six control sequences placed next to each other.
It is difficult to understand the code unless the programmer finds out which control sequences are functions and how many arguments each function absorbs.
To improve readability, \liii\ introduces a special naming convention where functions and variables are clearly distinguishable. In addition, programmers can gather more information from function and variable names such as the number of function arguments, the type of variables, and the scope of variables. 
%%%%%%%%%%%%%%%%%%%%%%%%%%%%%%%%%%%%PROTECT%%%%%%%%%%%%%%%%%%%%%%%%%%%%%%%%%%%%%
%%%%%%%%%%%%%%%%%%%%%%%%%%%%%%%%%%%%PROTECT%%%%%%%%%%%%%%%%%%%%%%%%%%%%%%%%%%%%%
%%%%%%%%%%%%%%%%%%%%%%%%%%%%%%%%%%%%PROTECT%%%%%%%%%%%%%%%%%%%%%%%%%%%%%%%%%%%%%
%%%%%%%%%%%%%%%%%%%%%%%%%%%%%%%%%%%%PROTECT%%%%%%%%%%%%%%%%%%%%%%%%%%%%%%%%%%%%%
%%%%%%%%%%%%%%%%%%%%%%%%%%%%%%%%%%%%PROTECT%%%%%%%%%%%%%%%%%%%%%%%%%%%%%%%%%%%%%
\subsection{Category Codes and Command Names}
%%%%%%%%%%%%%%%%%%%%%%%%%%%%%%%%%%%%PROTECT%%%%%%%%%%%%%%%%%%%%%%%%%%%%%%%%%%%%%
In \LaTeX, every input character can be classified into 16 categories. 
Each category can be identified by an integer ranging from 0 to 15, which is known as the \emph{category code}.
More details about category codes can be obtained from \cite{overleaf-catcode}.
We focus on one of the sixteen categories, which is known as ``letter''.
In most cases, command names in \LaTeX\ can only be made up of characters from the ``letter'' category.
Because the ``letter'' category only contains the lowercase and uppercase versions of the 26 English alphabets by default, command names are comprised of these 52 characters exclusively under the initial \LaTeX\ setup.
By extending the ``letter'' category, it is possible to add more permissible characters to command names.
For example, the \inlinestylea{\PY{k}{\PYZbs{}makeatletter}} command changes the category of \texttt{@} to ``letter'', which allows the character to be used in command names \cite{texse:makeatletter}. 
Many packages use the \texttt{@} character in their internal command names to maintain the integrity of defined commands.
Since typical users can only access commands whose names are made up of English alphabets, this technique can prevent internal commands from being overwritten accidentally.
The technique is also used in \liii\ to separate \liii\ from other \LaTeX\ programming conventions.
In \liii, two new characters are introduced into the command names, namely \texttt{\_} (underscore) and \texttt{:} (semicolon).
%%%%%%%%%%%%%%%%%%%%%%%%%%%%%%%%%%%%PROTECT%%%%%%%%%%%%%%%%%%%%%%%%%%%%%%%%%%%%%
The command \inlinestylea{\PY{k}{\PYZbs{}ExplSyntaxOn}} allows one to enter \liii\ programming mode.
It changes the category code of underscore and semicolon to ``letter''.
It also changes the category code associated with white space and line break characters so that they are ignored in \liii\ mode.
Therefore, a white space should be explicitly entered with \verb|~| or \verb|\ | in \liii\ mode.
The command \inlinestylea{\PY{k}{\PYZbs{}ExplSyntaxOff}} is used to exit \liii\ mode.
%%%%%%%%%%%%%%%%%%%%%%%%%%%%%%%%%%%%PROTECT%%%%%%%%%%%%%%%%%%%%%%%%%%%%%%%%%%%%%
%%%%%%%%%%%%%%%%%%%%%%%%%%%%%%%%%%%%PROTECT%%%%%%%%%%%%%%%%%%%%%%%%%%%%%%%%%%%%%
%%%%%%%%%%%%%%%%%%%%%%%%%%%%%%%%%%%%PROTECT%%%%%%%%%%%%%%%%%%%%%%%%%%%%%%%%%%%%%
%%%%%%%%%%%%%%%%%%%%%%%%%%%%%%%%%%%%PROTECT%%%%%%%%%%%%%%%%%%%%%%%%%%%%%%%%%%%%%
%%%%%%%%%%%%%%%%%%%%%%%%%%%%%%%%%%%%PROTECT%%%%%%%%%%%%%%%%%%%%%%%%%%%%%%%%%%%%%
%%%%%%%%%%%%%%%%%%%%%%%%%%%%%%%%%%%%PROTECT%%%%%%%%%%%%%%%%%%%%%%%%%%%%%%%%%%%%%
%%%%%%%%%%%%%%%%%%%%%%%%%%%%%%%%%%%%PROTECT%%%%%%%%%%%%%%%%%%%%%%%%%%%%%%%%%%%%%
\subsection{Names of Variables}\label{sec:name-of-variables}
%%%%%%%%%%%%%%%%%%%%%%%%%%%%%%%%%%%%PROTECT%%%%%%%%%%%%%%%%%%%%%%%%%%%%%%%%%%%%%
In \liii, the naming of variables and functions are different, which allows one to distinguish the two more easily. 
A \liii\ variable can be public or private depending on the semantics.
The name of each \liii\ variable is made up of four parts: \texttt{scope}, \texttt{module}, \texttt{description}, and \texttt{type}. 
We will describe the meaning of each part later.
%%%%%%%%%%%%%%%%%%%%%%%%%%%%%%%%%%%%PROTECT%%%%%%%%%%%%%%%%%%%%%%%%%%%%%%%%%%%%%
The name of a public variable should follow:
\begin{center}
\texttt{\string\<scope>\_<module>\_<description>\_<type>}
\end{center}
The name of a private variable should follow:
\begin{center}
\texttt{\string\<scope>\_\_<module>\_<description>\_<type>}
\end{center}
Private variables have an extra underscore between the \texttt{scope} part and the \texttt{module} part.
%%%%%%%%%%%%%%%%%%%%%%%%%%%%%%%%%%%%PROTECT%%%%%%%%%%%%%%%%%%%%%%%%%%%%%%%%%%%%%
The meaning of each part is shown as follows.
\begin{itemize}
\item \texttt{scope}: a single letter identifying the scope of the variable; see \cref{sec:var-scope} for more details
\begin{itemize}
\item \texttt{l}: local variable
\item \texttt{g}: global variable
\item \texttt{c}: constant
\end{itemize}
\item \texttt{module}: the name of the module where the variable is defined
\item \texttt{description}: the description of the variable
\item \texttt{type}: the variable type; some common types are shown in \cref{tbl:var-type}.
\end{itemize}
The \texttt{description} part can contain multiple segments separated by underscore(s). 
Some valid variables names are shown below.
\begin{itemize}
\item \inlinestylea{\PY{n+nv}{\PYZbs{}g\PYZus{}doc\PYZus{}variable\PYZus{}int}}
\item \inlinestylea{\PY{n+nv}{\PYZbs{}l\PYZus{}doc\PYZus{}bg\PYZus{}color\PYZus{}r\PYZus{}fp}}
\item \inlinestylea{\PY{n+nv}{\PYZbs{}l\PYZus{}doc\PYZus{}bg\PYZus{}color\PYZus{}g\PYZus{}fp}}
\item \inlinestylea{\PY{n+nv}{\PYZbs{}l\PYZus{}\PYZus{}mypkg\PYZus{}tmpa\PYZus{}seq}}
\item \inlinestylea{\PY{n+nv}{\PYZbs{}c\PYZus{}left\PYZus{}brace\PYZus{}str}}
\end{itemize}
%%%%%%%%%%%%%%%%%%%%%%%%%%%%%%%%%%%%PROTECT%%%%%%%%%%%%%%%%%%%%%%%%%%%%%%%%%%%%%
Since the category code of underscore has been changed to ``letter'' under \liii\ mode, it can no longer be used to denoted subscript in math equations.
One can use the predefined \liii\ constant \inlinestylea{\PY{n+nv}{\PYZbs{}c\PYZus{}math\PYZus{}subscript\PYZus{}token}} to represent subscripts when programming in \liii.
%%%%%%%%%%%%%%%%%%%%%%%%%%%%%%%%%%%%PROTECT%%%%%%%%%%%%%%%%%%%%%%%%%%%%%%%%%%%%%
\begin{table}[htpb]
\centering
\scriptsize
\begin{tabular}{>{\ttfamily\centering}p{0.15\linewidth}p{0.4\linewidth}}
\toprule
\multicolumn{1}{c}{Type} & Description\\ \midrule
clist & comma separated list\\
dim & dimension\\
fp & floating point number\\
int & integer\\
seq & sequence\\
str & string\\
tl & token list\\
bool & boolean\\
regex & regular expression\\
prop & property list\\
ior & IO read\\
iow & IO write\\ \bottomrule
\end{tabular}
\caption{Commonly used variable types and their descriptions. See ?? for more information about variable types.}
\label{tbl:var-type}
\end{table}
%%%%%%%%%%%%%%%%%%%%%%%%%%%%%%%%%%%%PROTECT%%%%%%%%%%%%%%%%%%%%%%%%%%%%%%%%%%%%%
\subsection{Names of Functions}
%%%%%%%%%%%%%%%%%%%%%%%%%%%%%%%%%%%%PROTECT%%%%%%%%%%%%%%%%%%%%%%%%%%%%%%%%%%%%%
To increase the readability of \liii\ code, a function name in \liii\ includes information about the arguments that it absorbs. The name of a public \liii\ function should follow:
\begin{center}
\texttt{\string\<module>\_<description>:<arg-spec>}
\end{center}
The name of a private \liii\ function should follow:
\begin{center}
\texttt{\string\_\_<module>\_<description>:<arg-spec>}
\end{center}
The semicolon is only used in \liii\ function names.
The meaning of each part is described as follows.
\begin{itemize}
\item \texttt{module}: the name of the module where the function is defined
\item \texttt{description}: the description of the function
\item \texttt{arg-spec}: the argument specifications--it is made up of zero or more English letters describing the type of every function argument, where each letter specifies the type of an argument; some common argument specification letters are shown in \cref{tbl:arg-spec-type}.
\end{itemize}
Some valid function names are shown below.
\begin{itemize}
\item \inlinestylea{\PY{k}{\PYZbs{}group\PYZus{}begin}\PY{n+nn}{:}}
\item \inlinestylea{\PY{k}{\PYZbs{}tl\PYZus{}put\PYZus{}right}\PY{n+nn}{:Nn}}
\item \inlinestylea{\PY{k}{\PYZbs{}tl\PYZus{}gput\PYZus{}right}\PY{n+nn}{:Nx}}
\item \inlinestylea{\PY{k}{\PYZbs{}\PYZus{}\PYZus{}doc\PYZus{}do\PYZus{}something}\PY{n+nn}{:Nnxx}}
\end{itemize}
%%%%%%%%%%%%%%%%%%%%%%%%%%%%%%%%%%%%PROTECT%%%%%%%%%%%%%%%%%%%%%%%%%%%%%%%%%%%%%
\begin{table}[htpb]
\centering
\scriptsize
\begin{tabular}{>{\ttfamily\centering}cm{0.6\linewidth}}
\toprule
\multicolumn{1}{c}{\texttt{arg-spec} Letter} & Description\\ \midrule
n & A token list argument\\
N & A single token argument\\
V & An argument passed by value\\
o & A token list argument expanded once\\
x & A token list argument expanded recursively\\
T & A true branch in conditional statements\\
F & A false branch in conditional statements\\
p & A parameter list\\
c & A token list argument that is fully expanded and treated as a command name\\ \bottomrule
\end{tabular}
\caption{Commonly used argument specification letters and their descriptions. More details are described in Section \textcolor{red}{???}.}
\label{tbl:arg-spec-type}
\end{table}
%%%%%%%%%%%%%%%%%%%%%%%%%%%%%%%%%%%%PROTECT%%%%%%%%%%%%%%%%%%%%%%%%%%%%%%%%%%%%%
%%%%%%%%%%%%%%%%%%%%%%%%%%%%%%%%%%%%PROTECT%%%%%%%%%%%%%%%%%%%%%%%%%%%%%%%%%%%%%
\subsection{The Use of \liii\ Naming Conventions}
%%%%%%%%%%%%%%%%%%%%%%%%%%%%%%%%%%%%PROTECT%%%%%%%%%%%%%%%%%%%%%%%%%%%%%%%%%%%%%
The \liii\ naming conventions are essentially suggestions. 
In reality, the \LaTeX\ compiler does not check if the code is written according to \liii\ naming conventions.
However, using the naming conventions can greatly improve the readability of one's code.
In Section \textcolor{red}{???}, we can also see that \liii's expansion control mechanism sometimes requires functions to be defined using \liii\ naming conventions.
Almost all \liii\ packages are written using the naming conventions.
Therefore, it is recommended that all \liii\ users follow the \liii\ naming conventions strictly.
%%%%%%%%%%%%%%%%%%%%%%%%%%%%%%%%%%%%PROTECT%%%%%%%%%%%%%%%%%%%%%%%%%%%%%%%%%%%%%
\section{Understanding \liii\ Documentation (I.1.2)}
%%%%%%%%%%%%%%%%%%%%%%%%%%%%%%%%%%%%PROTECT%%%%%%%%%%%%%%%%%%%%%%%%%%%%%%%%%%%%%
Currently, most information about \liii\ is compiled in \cite{l3interface}. 
It is mainly a technical documentation on \liii\ functions, scratch variables, and constants. 
We describe briefly how \cite{l3interface} documents these concepts.
\begin{itemize}
\item \liii\ functions: the majority of the contents in \cite{l3interface} are about \liii\ functions.
In \cite{l3interface}, each function is documented using a block similar to the one shown in  \cref{fig:l3-doc-func-example}.
The color background is added in the figure to help understand the block, it does not exist in the original document.
%%%%%%%%%%%%%%%%%%%%%%%%%%%%%%%%%%%%PROTECT%%%%%%%%%%%%%%%%%%%%%%%%%%%%%%%%%%%%%
The yellow part in \cref{fig:l3-doc-func-example} indicates the basic forms and the variants of the function. 
The first and third line in the yellow part are the basic forms for local and global assignment (see \cref{sec:var-scope}), respectively.
The second line shows the function variants for local assignment; the fourth line shows the function variants for global assignment.
More details about function variants will be introduced in Section \textcolor{red}{???}.
The cyan part in \cref{fig:l3-doc-func-example} shows the basic usage of the function, where the meaning of each function argument is described.
The magenta part in \cref{fig:l3-doc-func-example} shows the detailed description of the function.
%%%%%%%%%%%%%%%%%%%%%%%%%%%%%%%%%%%%PROTECT%%%%%%%%%%%%%%%%%%%%%%%%%%%%%%%%%%%%%
\item Scratch variables: similar to the C programming language \cite{ritchie1988c}, variables in \liii\ need to be defined before use. 
\liii\ has predefined a set of empty variables for convenience.
These variables are known as scratch variables.
In \liii, each package will usually define several scratch variables.
They are documented in specific sections, similar to the one demonstrated in \cref{fig:l3-scratch-var-example}.
The left hand side of \cref{fig:l3-scratch-var-example} shows the predefined scratch variables; the right hand side shows the description of the variables.
It is recommended not to use scratch variables, especially in large projects or generic packages.
This reduces the chances of collision.
%%%%%%%%%%%%%%%%%%%%%%%%%%%%%%%%%%%%PROTECT%%%%%%%%%%%%%%%%%%%%%%%%%%%%%%%%%%%%%
\item Constants: \liii\ defines a large number of constants such as the value of $\pi$, the value of $e$, special characters, etc. 
They are documented in specific sections, similar to the one demonstrated in \cref{fig:l3-constant-example}.
The left hand side of \cref{fig:l3-constant-example} shows the command name of the constant; the right hand side shows the description of the constant.
%%%%%%%%%%%%%%%%%%%%%%%%%%%%%%%%%%%%PROTECT%%%%%%%%%%%%%%%%%%%%%%%%%%%%%%%%%%%%%
\end{itemize}
%%%%%%%%%%%%%%%%%%%%%%%%%%%%%%%%%%%%PROTECT%%%%%%%%%%%%%%%%%%%%%%%%%%%%%%%%%%%%%
\begin{figure*}[htpb]
\centering
\mdfdefinestyle{plainmdbox}{topline=false,leftline=false,rightline=false,bottomline=false}
\begin{minipage}[t]{0.52\linewidth}
\begin{mdframed}[backgroundcolor=yellow!50, style=plainmdbox]
\begin{tabular}{l}
\toprule
\verb|\tl_set:Nn|\\
{\color{gray}\verb|\tl_set|}\verb!:(NV|Nv|No|Nf|Nx|cn|cV|cv|co|cf|cx)! \\
\verb|\tl_gset:Nn| \\
{\color{gray}\verb|\tl_gset|}\verb!:(NV|Nv|No|Nf|Nx|cn|cV|cv|co|cf|cx)! \\ \bottomrule
\end{tabular}
\end{mdframed}
\end{minipage}
\begin{minipage}{0.4\linewidth}
\vspace*{3em}
\begin{mdframed}[backgroundcolor=cyan!50, style=plainmdbox]
\verb|\tl_set:Nn| {\ttfamily<\textsl{tl var}> \{<\textsl{tokens}>\}}
\end{mdframed}
\end{minipage}\\
\begin{minipage}[t]{0.15\linewidth}
\phantom{}
\end{minipage}
\begin{minipage}{0.82\linewidth}
\begin{mdframed}[backgroundcolor=magenta!50, style=plainmdbox]
Sets <\textit{tl var}> to contain <\textit{tokens}>, removing any previous content from the variable.
\end{mdframed}
\end{minipage}
\caption{An example of function documentation excerpted from Section IV.15.2 of \protect\cite{l3interface}. 
Color background is added in this figure to help understand the documentation.}
\label{fig:l3-doc-func-example}
\end{figure*}
%%%%%%%%%%%%%%%%%%%%%%%%%%%%%%%%%%%%PROTECT%%%%%%%%%%%%%%%%%%%%%%%%%%%%%%%%%%%%%
\begin{figure*}
\centering
\begin{minipage}[t]{0.15\linewidth}
\begin{tabular}{l}
\toprule
\verb|\l_tmpa_tl|\\
\verb|\l_tmpb_tl| \\ \bottomrule
\end{tabular}
\end{minipage}
\begin{minipage}[t]{0.83\linewidth}
\vspace*{-1em}
Scratch token lists for local assignment.
These are never used by the kernel code, and so are safe for use with any \liii--defined function.
However, they may be overwritten by other non-kernel code and so should only be used for short-term storage.
\end{minipage}
\caption{An example of scratch variables excerpted from Section IV.15.8 of \protect\cite{l3interface}.}
\label{fig:l3-scratch-var-example}
\end{figure*}
%%%%%%%%%%%%%%%%%%%%%%%%%%%%%%%%%%%%PROTECT%%%%%%%%%%%%%%%%%%%%%%%%%%%%%%%%%%%%%
\begin{figure*}
\centering
\begin{minipage}[t]{0.15\linewidth}
\begin{tabular}{l}
\toprule
\verb|\c_pi_fp|\\ \bottomrule
\end{tabular}
\end{minipage}
\begin{minipage}[t]{0.83\linewidth}
The value of $\pi$. This can be input directly in a floating point expression as \texttt{pi}.
\end{minipage}
\caption{An example of scratch variables excerpted from Section IV.28.6 of \protect\cite{l3interface}.}
\label{fig:l3-constant-example}
\end{figure*}
%%%%%%%%%%%%%%%%%%%%%%%%%%%%%%%%%%%%PROTECT%%%%%%%%%%%%%%%%%%%%%%%%%%%%%%%%%%%%%
%%%%%%%%%%%%%%%%%%%%%%%%%%%%%%%%%%%%PROTECT%%%%%%%%%%%%%%%%%%%%%%%%%%%%%%%%%%%%%
\section{Variables and Functions}
%%%%%%%%%%%%%%%%%%%%%%%%%%%%%%%%%%%%PROTECT%%%%%%%%%%%%%%%%%%%%%%%%%%%%%%%%%%%%%
%%%%%%%%%%%%%%%%%%%%%%%%%%%%%%%%%%%%PROTECT%%%%%%%%%%%%%%%%%%%%%%%%%%%%%%%%%%%%%
%%%%%%%%%%%%%%%%%%%%%%%%%%%%%%%%%%%%PROTECT%%%%%%%%%%%%%%%%%%%%%%%%%%%%%%%%%%%%%
\appendix
%%%%%%%%%%%%%%%%%%%%%%%%%%%%%%%%%%%%PROTECT%%%%%%%%%%%%%%%%%%%%%%%%%%%%%%%%%%%%%
%%%%%%%%%%%%%%%%%%%%%%%%%%%%%%%%%%%%PROTECT%%%%%%%%%%%%%%%%%%%%%%%%%%%%%%%%%%%%%
\section{Scope of Variables}\label{sec:var-scope}
%%%%%%%%%%%%%%%%%%%%%%%%%%%%%%%%%%%%PROTECT%%%%%%%%%%%%%%%%%%%%%%%%%%%%%%%%%%%%%
In \liii, the scope of value assignment is usually local.
\exfullref{ex:group-demo} shows the difference between local and global assignment.
In this example, \lref{ex:group-demo}{4}[9] are enclosed in a group.
Inside the group, the value of \inlinestylea{\PY{n+nv}{\PYZbs{}l\PYZus{}tmpa\PYZus{}tl}} is locally assigned with \inlinestylea{\PY{k}{\PYZbs{}tl\PYZus{}set}\PY{n+nn}{:Nn}}, while the value of \inlinestylea{\PY{n+nv}{\PYZbs{}g\PYZus{}tmpa\PYZus{}tl}} is globally assigned with \inlinestylea{\PY{k}{\PYZbs{}tl\PYZus{}gset}\PY{n+nn}{:Nn}}.
Since the value assignment of \inlinestylea{\PY{n+nv}{\PYZbs{}l\PYZus{}tmpa\PYZus{}tl}} is local, the scope of the assignment is constrained within the group.
Therefore, when we use its value outside the group, we get the old value.
In contrast, the value assignment of \inlinestylea{\PY{n+nv}{\PYZbs{}g\PYZus{}tmpa\PYZus{}tl}} is able to go beyond group boundaries.
%%%%%%%%%%%%%%%%%%%%%%%%%%%%%%%%%%%%PROTECT%%%%%%%%%%%%%%%%%%%%%%%%%%%%%%%%%%%%%
\def\CurrentEnvName{latexsample}
\begin{filecontents}[force,noheader]{\jobname-lst-raw.vrb}
\ExplSyntaxOn
\tl_set:Nn \l_tmpa_tl {old}
\tl_gset:Nn \g_tmpa_tl {old}
\group_begin:
\tl_set:Nn \l_tmpa_tl {new} % set value locally


% comments should be here


\tl_gset:Nn \g_tmpa_tl {new} % set value globally
\par In~group:~\tl_use:N \l_tmpa_tl
\par In~group:~\tl_use:N \g_tmpa_tl
\group_end:
\par Outside~group:~\tl_use:N \l_tmpa_tl
\par Outside~group:~\tl_use:N \g_tmpa_tl
\ExplSyntaxOff
\end{filecontents}
\begin{filecontents}[force,noheader]{\jobname-lst-hl.vrb}
\begin{Verbatim}[commandchars=\\\{\}]
\PY{k}{\PYZbs{}ExplSyntaxOn}
\PY{k}{\PYZbs{}tl\PYZus{}set}\PY{n+nn}{:Nn} \PY{n+nv}{\PYZbs{}l\PYZus{}tmpa\PYZus{}tl} \PY{n+nb}{\PYZob{}}old\PY{n+nb}{\PYZcb{}}
\PY{k}{\PYZbs{}tl\PYZus{}gset}\PY{n+nn}{:Nn} \PY{n+nv}{\PYZbs{}g\PYZus{}tmpa\PYZus{}tl} \PY{n+nb}{\PYZob{}}old\PY{n+nb}{\PYZcb{}}
\PY{k}{\PYZbs{}group\PYZus{}begin}\PY{n+nn}{:}
\PY{k}{\PYZbs{}tl\PYZus{}set}\PY{n+nn}{:Nn} \PY{n+nv}{\PYZbs{}l\PYZus{}tmpa\PYZus{}tl} \PY{n+nb}{\PYZob{}}new\PY{n+nb}{\PYZcb{}} \PY{c}{\PYZpc{} set value locally}


\PY{c}{\PYZpc{} comments should be here}


\PY{k}{\PYZbs{}tl\PYZus{}gset}\PY{n+nn}{:Nn} \PY{n+nv}{\PYZbs{}g\PYZus{}tmpa\PYZus{}tl} \PY{n+nb}{\PYZob{}}new\PY{n+nb}{\PYZcb{}} \PY{c}{\PYZpc{} set value globally}
\PY{k}{\PYZbs{}par} In\PYZti{}group:\PYZti{}\PY{k}{\PYZbs{}tl\PYZus{}use}\PY{n+nn}{:N} \PY{n+nv}{\PYZbs{}l\PYZus{}tmpa\PYZus{}tl}
\PY{k}{\PYZbs{}par} In\PYZti{}group:\PYZti{}\PY{k}{\PYZbs{}tl\PYZus{}use}\PY{n+nn}{:N} \PY{n+nv}{\PYZbs{}g\PYZus{}tmpa\PYZus{}tl}
\PY{k}{\PYZbs{}group\PYZus{}end}\PY{n+nn}{:}
\PY{k}{\PYZbs{}par} Outside\PYZti{}group:\PYZti{}\PY{k}{\PYZbs{}tl\PYZus{}use}\PY{n+nn}{:N} \PY{n+nv}{\PYZbs{}l\PYZus{}tmpa\PYZus{}tl}
\PY{k}{\PYZbs{}par} Outside\PYZti{}group:\PYZti{}\PY{k}{\PYZbs{}tl\PYZus{}use}\PY{n+nn}{:N} \PY{n+nv}{\PYZbs{}g\PYZus{}tmpa\PYZus{}tl}
\PY{k}{\PYZbs{}ExplSyntaxOff}
\end{Verbatim}
\end{filecontents}
%%%%%%%%%%%%%%%%%%%%%%%%%%%%%%%%%%%%PROTECT%%%%%%%%%%%%%%%%%%%%%%%%%%%%%%%%%%%%%
In \cref{sec:name-of-variables}, it is described that the scope of variables should be encoded as the first letter (prefix) of variable names.
If a variable is to be assigned using \verb|set| functions, we can consider it to be local and use \verb|l| as the prefix.
If a variable is to be assigned using \verb|gset| functions, we can consider it to be global and use \verb|g| as the prefix.
%%%%%%%%%%%%%%%%%%%%%%%%%%%%%%%%%%%%PROTECT%%%%%%%%%%%%%%%%%%%%%%%%%%%%%%%%%%%%%
\bibliographystyle{tugboat}
\bibliography{main.bib}
%%%%%%%%%%%%%%%%%%%%%%%%%%%%%%%%%%%%PROTECT%%%%%%%%%%%%%%%%%%%%%%%%%%%%%%%%%%%%%
\makesignature
%%%%%%%%%%%%%%%%%%%%%%%%%%%%%%%%%%%%PROTECT%%%%%%%%%%%%%%%%%%%%%%%%%%%%%%%%%%%%%
\end{document}