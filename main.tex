% !BIB TS-program = bibtex
% !TeX TS-program = pdflatex
\documentclass{ltugboat}

\usepackage[T1]{fontenc}
\usepackage{lmodern}
\usepackage{amsmath, amssymb}

\DeclareFontFamily{T1}{DejaVuSansMono-TLF}{\hyphenchar\font-1}
\DeclareFontShape{T1}{DejaVuSansMono-TLF}{m}{n}{<-> s*[0.75]DejaVuSansMono-tlf-t1}{}
\DeclareFontShape{T1}{DejaVuSansMono-TLF}{m}{it}{<-> s*[0.75]DejaVuSansMono-Oblique-tlf-t1}{}
\DeclareFontShape{T1}{DejaVuSansMono-TLF}{b}{n}{<-> s*[0.75]DejaVuSansMono-Bold-tlf-t1}{}
\DeclareFontShape{T1}{DejaVuSansMono-TLF}{b}{it}{<-> s*[0.75]DejaVuSansMono-BoldOblique-tlf-t1}{}

\DeclareFontFamily{T1}{lmtt}{\hyphenchar\font-1}
\DeclareFontShape{T1}{lmtt}{m}{n}{<-> s*[0.9]ec-lmtt10}{}
\DeclareFontShape{T1}{lmtt}{m}{it}{<-> s*[0.9]ec-lmtti10}{}
\DeclareFontShape{T1}{lmtt}{m}{sl}{<-> s*[0.9]ec-lmtto10}{}
\usepackage[english]{babel}
\usepackage{metalogo}
\usepackage{tcolorbox}
\usepackage{booktabs}
\usepackage{microtype}
\usepackage{expl3}
\usepackage{mdframed}
\usepackage{chngcntr}
\usepackage{etoolbox}
\usepackage{adjustbox}
\usepackage{supertabular}
\usepackage{makecell}
\usepackage{tikz}
\usepackage{ragged2e}
\usepackage{float}
\usepackage{array}
\usepackage{xcolor}

\ExplSyntaxOn
\file_if_exist:nTF{USE_SHELL_ESCAPE}
{
    \file_if_exist:nTF{MINTED_FINALIZE}
    {
        \iow_term:n {=====~Document~is~compiled~in~finalization~mode~=====}
        \usepackage[finalizecache]{minted}
    }
    {
        \iow_term:n {=====~Document~is~compiled~in~development~mode~=====}
        \usepackage{minted}
    }
}
{
    \iow_term:n {=====~Document~is~compiled~in~submission~mode~=====}
    \usepackage[frozencache]{minted}
}
\ExplSyntaxOff


%\usepackage[breaklinks,colorlinks,pdfa]{hyperref}
\usepackage[breaklinks]{hyperref}
\usepackage[capitalise]{cleveref}


\tcbuselibrary{listings, minted, skins, breakable, xparse, hooks}

% a quick command for LaTeX3
\newcommand*{\liii}{\LaTeX3}

\renewcommand{\MintedPygmentize}{./my_pygmentize}
% other code highliting setup
\newmintinline[inltex]{tex_lexer.py:Tex3Lexer}{breaklines, breakanywhere, fontfamily=DejaVuSansMono-TLF}
\newmintinline[inlpy]{python}{fontfamily=DejaVuSansMono-TLF}
\newmintinline[inlpl]{text}{}

% set the style of the list of examples
% https://tex.stackexchange.com/questions/86711/tcolorbox-list-of-listings
\newcommand{\ListOfCodeExampleName}{List of Examples}
\makeatletter
\newcommand{\ListOfCodeExample}{\section*{\ListOfCodeExampleName}\@starttoc{CodeExample}}
\makeatother

% the counter for numbered code examples
\newcounter{codeexample}
\counterwithin{codeexample}{section}

%%%%%%%%%%%%%%%%%%%%%%%%%%%%%%%%%%%%%%%%%%%%%%%%%%%%%%%%%%%%%%%%%%
% this segment of code is to implement arbitrary code
% cross-referencing within the article
%%%%%%%%%%%%%%%%%%%%%%%%%%%%%%%%%%%%%%%%%%%%%%%%%%%%%%%%%%%%%%%%%%
\makeatletter
\ExplSyntaxOn
\bool_new:N \g_lst_line_ref_bool
\tl_new:N \g_lst_label_tl
\tl_new:N \l_lst_tmpa_tl
\tl_new:N \l_lst_tmpb_tl
\prop_new:N \g_lst_index_prop

\cs_set:Npn \__lst_write_aux:n #1 {
    \immediate\write\@auxout{#1}
}

\newcommand{\EnableLstRef}{\bool_gset_true:N \g_lst_line_ref_bool}
\newcommand{\DisableLstRef}{\bool_gset_false:N \g_lst_line_ref_bool}
\newcommand{\SetLstRefLabel}[1]{\tl_gset:Nn \g_lst_label_tl {#1}}
\newcommand{\LstPutLookup}[1]{
    \tl_set:Nn \l_lst_tmpb_tl {
       \string\expandafter\string\gdef\string\csname\space @@lst-lookup-#1\string\endcsname{\TheLstCounter}
    }
    \exp_args:Nx \__lst_write_aux:n \l_lst_tmpb_tl
}
\newcommand{\LstUseLookup}[1]{
    \cs_if_exist:cTF {@@lst-lookup-#1} {
        \use:c {@@lst-lookup-#1}
    } {
        ?
    }
}


\cs_set:Npn \__lst_render_line: {
    \textcolor[rgb]{0.5,0.5,0.5}{
    \ttfamily\scriptsize
    \arabic{FancyVerbLine}
    }
}

\cs_set:Npn \__lst_make_hypertarget:nn #1 #2{
   \raisebox{1em}{\hypertarget{#1}{}}#2 
}

\renewcommand{\theFancyVerbLine}{
    \bool_if:NTF \g_lst_line_ref_bool {
        \tl_set:Nx \l_lst_tmpa_tl {\arabic{FancyVerbLine}}
        \exp_args:Nx \__lst_make_hypertarget:nn {\g_lst_label_tl-\l_lst_tmpa_tl}{
            \__lst_render_line:
        }
    }{
        \__lst_render_line:
    }

}

\int_new:N \g_lst_counter_int
\newcommand{\AddLstCounter}{\int_gincr:N \g_lst_counter_int}
\newcommand{\TheLstCounter}{\int_use:N \g_lst_counter_int}

% user command to cross reference a line in the lising
% #1: listing name
% #2: line number
% #3 (optional): would output a range of lines from #2 to #3 if present
\DeclareDocumentCommand{\lref}{mmo}{
    \IfValueTF{#3}{Lines}{Line}~\hyperlink{#1-#2}{
        \LstUseLookup{#1}:
        \IfValueTF{#3}{#2-#3}{#2}
    }
}

% user command to cross referenec a listing
\DeclareDocumentCommand{\lstref}{m}{
    Listing~\hyperlink{#1-1}{
        \LstUseLookup{#1}
    }
}

\ExplSyntaxOff
\makeatother
%%%%%%%%%%%%%%%%%%%%%%%%%%%%%%%%%%%%%%%%%%%%%%%%%%%%%%%%%%%%%%%%%%
%%%%%%%%%%%%%%%%%%%%%%%%%%%%%%%%%%%%%%%%%%%%%%%%%%%%%%%%%%%%%%%%%%

% basic style for examples
\tcbset{
  codesample/.style={
    enhanced,
    breakable,
    beforeafter skip=3ex,
    listing engine=minted,
    minted language=tex_lexer.py:Tex3Lexer,
    fontlower=\sffamily\small,
    minted options={
      fontsize=\fontsize{9}{9},
      autogobble,
      breaklines,
      obeytabs,
      tabsize=2,
      linenos,
      numbersep=2mm,
      xleftmargin=4mm,
      fontfamily=DejaVuSansMono-TLF
    },
    fonttitle=\normalfont,
    colback=white,
    colframe=black!60,
    boxrule=0.8pt,
    left=1mm,
    right=1mm,
    top=0.5mm,
    bottom=0.5mm,
    before upper pre={\AddLstCounter},
    overlay unbroken and last={
        \node[font=\sffamily\bfseries\scriptsize, anchor=south east] at (frame.south east) {\color[rgb]{0.5,0.5,0.5} \TheLstCounter};
    }
  }
}

%%%%%%%%%%%%%%%%%%%%%%%%%%%%%%%%%%%%%%%%%%%%%%%%%%%%%%%%%%%%%%%%%%
% this segment of code is to implement code exporting
% if \ExportExamplestrue is called, then the source code of each
% example will be output to /examples
%%%%%%%%%%%%%%%%%%%%%%%%%%%%%%%%%%%%%%%%%%%%%%%%%%%%%%%%%%%%%%%%%%
\newif\ifExportExamples

\ifExportExamples

% define our custom VerbatimOut like environment for latexsample
\makeatletter

\ExplSyntaxOn

\cs_set:Npn \__doc_write:Nn #1#2 {
    \immediate\write#1{#2}
}
\cs_generate_variant:Nn \__doc_write:Nn {Nx}
\cs_set_eq:NN \MyWriteA \__doc_write:Nn
\cs_set_eq:NN \MyWriteB \__doc_write:Nx
\cs_set_eq:NN \PCTSGN \c_percent_str

\cs_set:Npn \__doc_open_out:Nn #1#2 {
    \immediate\openout#1 #2\relax
}
\cs_generate_variant:Nn \__doc_open_out:Nn {Nx}
\cs_set_eq:NN \MyOpenOut \__doc_open_out:Nx

\tl_new:N \l_doc_tmpa_tl
\tl_new:N \ExampleFilename

\cs_set:Npn \UpdateExampleFilename {
    \tl_set_eq:NN \l_doc_tmpa_tl \thecodeexample
    \regex_replace_all:nnN {\.} {-} \l_doc_tmpa_tl
    \tl_set_eq:NN \ExampleFilename \l_doc_tmpa_tl
}

\ExplSyntaxOff

\newwrite\FV@OutFileA
\newwrite\FV@OutFileB

\def\MyProcessLine#1{%
    \immediate\write\FV@OutFileA{#1}%
    \immediate\write\FV@OutFileB{#1}%
}

\def\latexsample{\FV@Environment{}{latexsample}}
\def\FVB@latexsample#1{%
  \@bsphack
  \begingroup
    \FV@UseKeyValues
    \FV@DefineWhiteSpace
    \def\FV@Space{\space}%
    \FV@DefineTabOut
    \let\FV@ProcessLine\MyProcessLine
    % increment example counter
    \refstepcounter{codeexample}
    \UpdateExampleFilename
    %\def\FV@ProcessLine{\immediate\write\FV@OutFile}%
    % copy1: this is to be read back and compiled in the document
    \MyOpenOut\FV@OutFileA{temp-example.vrb}
    %\immediate\openout\FV@OutFileA temp-example.vrb\relax
    % copy2: this is to be saved in examples folder
    %\immediate\openout\FV@OutFileB temp-example2.vrb\relax
    \MyOpenOut\FV@OutFileB{examples/example-\ExampleFilename.tex}
    % write the comment in the example
    \MyWriteB\FV@OutFileB{\PCTSGN\PCTSGN\space Example \thecodeexample: #1}
    % save #1
    \gdef\TempExampleTitle{#1}
    \let\FV@FontScanPrep\relax
%% DG/SR modification begin - May. 18, 1998 (to avoid problems with ligatures)
    \let\@noligs\relax
%% DG/SR modification end
    \FV@Scan}
\def\FVE@latexsample{%
\immediate\closeout\FV@OutFileA%
\immediate\closeout\FV@OutFileB%
\endgroup\@esphack%
% input listing
\inputlatexsample{\TempExampleTitle}{temp-example.vrb}
}

\DefineVerbatimEnvironment{latexsample}{latexsample}{}

\makeatother


\newtcbinputlisting{\inputlatexsample}[2]{
  codesample,
  title=\GenExampleTitle{#1},
  listing file={#2}
}

\newcommand{\GenExampleTitle}[1]{%
    %\refstepcounter{codeexample}
    \hspace*{0.5em}
    Example \thecodeexample:~#1
    \addcontentsline{CodeExample}{subsection}{\protect\numberline{\thecodeexample}#1}
}

%TODO: fix the environment when exporting!

\else

\newcommand{\GenExampleTitle}[2]{%
    \refstepcounter{codeexample}
    \IfValueT{#2}{\label{#2}}
    \hspace*{0.5em}
    Example \thecodeexample:~#1
    \addcontentsline{CodeExample}{subsection}{\protect\numberline{\thecodeexample}#1}
}

\DeclareTCBListing{latexsample}{m!o}{
    codesample,
    title=\GenExampleTitle{#1}{#2},
    IfValueTF={#2}
    {
        before upper app={\EnableLstRef\SetLstRefLabel{#2}\LstPutLookup{#2}},
        after app={\DisableLstRef}
    }
    {}
}

%\newtcblisting{latexsample}[1]{
%  codesample,
%  title=\GenExampleTitle{#1}
%}

\fi


%%%%%%%%%%%%%%%%%%%%%%%%%%%%%%%%%%%%%%%%%%%%%%%%%%%%%%%%%%%%%%%%%%
%%%%%%%%%%%%%%%%%%%%%%%%%%%%%%%%%%%%%%%%%%%%%%%%%%%%%%%%%%%%%%%%%%



% listing only example
\DeclareTCBListing{latexsample**}{!o}{
    codesample,
    listing only,
    IfValueTF={#1}
    {
        before upper app={\EnableLstRef\SetLstRefLabel{#1}\LstPutLookup{#1}},
        after app={\DisableLstRef}
    }
    {}
}

% anonymous example
\DeclareTCBListing{latexsample*}{!o}{
    codesample,
    IfValueTF={#1}
    {
        before upper app={\EnableLstRef\SetLstRefLabel{#1}\LstPutLookup{#1}},
        after app={\DisableLstRef}
    }
    {}
}

%\newtcblisting{latexsample*}{
%  codesample
%}

% listing of any language
\newtcblisting{codesample}[1]{
  codesample,
  listing only,
  minted language=#1
}



\makeatletter
% better URL line breaking
\g@addto@macro{\UrlBreaks}{\UrlOrds}

%\newcommand{\inlcolorbox}[2]{
%    \bgroup
%    %\setlength{\fboxsep}{0pt}
%    \colorbox{blue!30}{#2}
%    \egroup
%}

%\patchcmd{\minted@inputpyg@inline}{\colorbox}{\inlcolorbox}{}{\GenericError{}{cannot patch minted}{}{}}
%\let\old@minted@inputpyg@inline\minted@inputpyg@inline
%\renewcommand{cmd}{def}

\crefname{codeexample}{Example}{Examples}
\newcommand{\exfullref}[1]{Example \ref{#1} (\lstref{#1})}

\makeatother
\UseOverleafMode
\ExplSyntaxOn

\makeatletter

% temporary fix to support custom lexer...
\tl_set:Nx \minted@optlistcl@g
{
    \exp_not:V \minted@optlistcl@g \space -x \space 
}

\bool_if:NTF \g_doc_overleaf_mode_bool
{
    % overleaf mode
    \iow_term:n {=====~Document~is~compiled~in~Overleaf~development~mode~=====}
    \usepackage{minted}
}
{
    % Makefile mode
    % \renewcommand{\MintedPygmentize}{./my_pygmentize}
    
    \file_if_exist:nTF{USE_SHELL_ESCAPE}
    {
        \file_if_exist:nTF{MINTED_FINALIZE}
        {
            \iow_term:n {=====~Document~is~compiled~in~finalization~mode~=====}
            \usepackage[finalizecache,cachedir={minted-frozen}]{minted}
        }
        {
            \iow_term:n {=====~Document~is~compiled~in~development~mode~=====}
            \usepackage{minted}
        }
    }
    {
        \iow_term:n {=====~Document~is~compiled~in~submission~mode~=====}
        \usepackage[frozencache,cachedir={minted-frozen}]{minted}
    }
}
\makeatother
\ExplSyntaxOff

\title{A \LTT{} programming tutorial}
\author{Ziyue Xiang}
\address{Purdue University}
\personalURL{https://www.alanshawn.com}
\ORCID{0000-0001-6054-5801}


\ExportExamples
%\UseMintedHighlighting

\begin{document}

\begin{abstract}
\LT{} is widely used to generate high quality documents.
It provides not only control over document content and style, but also the logic behind the document generation process.
For many users, knowledge on the programming side of \LT{} can facilitate document generation.
In this tutorial, we introduce programming in \LT{}.
We focus on \LTT{}--a set of modern programming interfaces in \LT{} to help users write general purpose programs.
\LTT{} contains functionalities such as boolean logic, integer/floating point arithmetic, data structures, and regular expressions.
This is an example based \LTT{} tutorial that introduces readers to the commonly used components of \LTT{}.
\end{abstract}

\maketitle

\tableofcontents
\ListOfCodeExample


\section{Introduction}

\LT{} is widely known as a typesetting language.
Many \LT{} users are familiar with \LT{}'s interface to control the document's content and style.
It is often overlooked that \LT{} also provides generic programming interface to control the logic of the document generation process.
In this tutorial, we introduce \LTT{}--a set of modern programming interfaces provided by \LT{} to help users implement various general purpose algorithms.

This is an example based tutorial. 
Examples are referenced using the following notation: \cref{ex:groups}.
A line in an example is referenced using the following notation: \lref{ex:groups}{1}.
Here, the @ symbol is used to separate the line number and the example number.
A range of lines in an example is referenced using the following notation: \lref{ex:groups}{1-5}.

\subsection{Important Concepts in \LT{}}\label{sec:important-concepts}

\LT{} is a programming language. 
It is Turing complete just like popular generic programming languages such as Python and C++, which means that they all have the same problem solving capabilities.
\LT{} provides support for boolean logic, arithmetic, data structures, and string processing.
Before introducing how to program in \LT{}, we want to describe three concepts that every \LT{} programmer should know.
Some of these concepts are unique to \LT{} and cannot be found in popular programming languages.

The first concept is category code.
In many existing languages, valid function/variable names and meanings of characters are predefined and cannot be changed.
In \LT{}, the user has the freedom to change them.
For example, normally \verb|\| is used to start a control sequence, \verb|$| is used to toggle math mode, \verb|%| is used for comments, and Latin alphabets are used as control sequence names.
However, with several lines of code, one can change the configurations so that \verb|@| is used to start a control sequence, \verb|+| is used to toggle math mode, \verb|#| is used for comments, and both Latin alphabets and Arabic numerals are used as control sequence names.
%To understand how this is possible, we need to introduce the concept of \emph{category code}.
This is made possible by using category code.
Category code is a hidden attribute associated with every character.
In \cref{tab:cat-code}, we show the 16 category codes in \LT{} and example characters with given category codes in default \LT{} configuration.
In \LT{}, the meaning of each character is determined by its category code instead of its character code. 
% For example, any character with category code 7 can be used to denote superscript.
% In default \LT{} configuration, the character \verb|^| is set to have character code 7 so that it can be used for superscript.
% The category code associated with each character can be changed. 
% A user can generate a character \verb|^| with category code 8, which means that it can be used to denote subscript.
In \cref{ex:category-code}, we show the category code of three characters: \verb|^|, \verb|_|, and \verb|*|.
The category code of \verb|^| and \verb|_| are 7 and 8, respectively. 
That is the reason why they can be used to denoted superscript and subscript.
In \lref{ex:category-code}{8}, we change the category code of \verb|*| to 7.
Notice how it triggers subscript in math equations afterwards.
\begin{latexsample}[examplelabel={ex:category-code},exampletitle={Category Code}]
% show the category code of some characters
\par The category code of \^{} is \the\catcode`^
\par The category code of \_{} is \the\catcode`_
\par The category code of * is \the\catcode`*
% a match equation
\par $a*b~a^b~a_b$
% change the category code of * to 7
\catcode`*=7
% see what changes in the equation
\par $a*b~a^b~a_b$
\end{latexsample}
\noindent 
%Category code makes \LT{} different from many existing programming languages.
It is worth noting that in \LT{}, two characters are considered equal if and only if their character codes are the same and their category codes are the same.


\begin{table}[tb]
  \centering
  \footnotesize
  \begin{tabular}{|c|>{\centering\arraybackslash}m{0.3\linewidth}|>{\centering\arraybackslash}m{0.3\linewidth}|}
    \hline
    \makecell{Category\\Code} & Description & \makecell{Example\\Character(s)} \\ \hline
    0 & Start a control sequence  & \verb|'\'| \\ \hline
    1 & Start a group & \verb|'{'| \\ \hline
    2 & End a group & \verb|'}'| \\ \hline
    3 & Toggle math mode & \verb|'$'| \\ \hline
    4 & Alignment tab & \verb|'&'| \\ \hline
    5 & End of line & ASCII(13) \\ \hline
    6 & Macro parameter & \verb|'#'| \\ \hline
    7 & Superscript & \verb|'^'| \\ \hline
    8 & Subscript & \verb|'_'| \\ \hline
    9 & Ignored character & ASCII(0) \\ \hline
    10 & White space (spacer) & ASCII(32), ASCII(9) \\ \hline
    11 & Letter & \verb|'a'|, \ldots, \verb|'z'|, \verb|'A'|, \ldots, \verb|'Z'|\\ \hline
    12 & Other & \verb|'0'|, \ldots, \verb|'9'|, \verb|';'|, \verb|','|, \verb|'?'|, \ldots\\ \hline
    13 & Active (used as commands) & \verb|'~'|, \ldots\\ \hline
    14 & Comment & \verb|%| \\ \hline
    15 & Invalid & ASCII(127) \\ \hline
  \end{tabular}
  \caption{List of the 16 category codes in \LT{} and example characters with given category codes under standard \LT{} configuration. Printable characters are surrounded with single quotes; unprintable characters are represented using ASCII codes.}
  \label{tab:cat-code}
\end{table}


The second concept is group.
Groups in \LT{} are similar to scopes in other programming languages.
By default, almost all changes (e.g., macro definition, category code assignment, font selection) in \LT{} are local, which means they only apply to the current group.
These changes are not visible outside the group.
In \LT{}, one way to create a new group is to use braces (\verb|{}|).
In \cref{ex:groups}, we show the effect of groups.
On \lref{ex:groups}{2}, we define the macro \inltex|\abc|.
On \lref{ex:groups}{6}, we update the definition of \inltex|\abc|.
Notice how the updated definition only applies to the group enclosed by \lref{ex:groups}{4-8}.
Similar principle is also reflected by fonts: the \inltex|\normalfont| command on \lref{ex:groups}{5} only affects the contents within the group.
\begin{latexsample}[examplelabel={ex:groups},exampletitle={Groups}]
\itshape
\def\abc{abc }%
\abc%
{%
  \normalfont%
  \def\abc{def }%
  \abc%
}%
\abc%
\end{latexsample}
\noindent In \LT{}, the user can also make changes that are global, meaning that these changes will be visible everywhere.
It is vital for \LT{} users to determine if a change should be local or global.


The third concept is macro expansion control.
Fundamentally, \LT{} is a simple macro language that works by replacing patterns according to a set of defined macro functions.
The \LT{} compiler will not attempt to generate an Abstract Syntax Tree (AST) for the \LT{} source, which means it has very little understanding over its input compared to other languages' compilers that rely on ASTs.
One of the main consequences of such simplification is \LT{} compilers are less powerful in terms of function calls, because it has limited insight over the function arguments.
In \cref{ex:trad-expan-ctrl}, suppose we have a function \inltex|\cmda| that scans each character in its argument \verb|#1| and apply some character-wise operation. 
In \inltex|\cmda|, we use the \inltex|\detokenize| command to output the verbatim of its argument without parsing.
Because it is common to store characters in a control sequence, let us observe what happens when \inltex|\cmda| is called with a control sequence \inltex|\vala| as its argument (\lref{ex:trad-expan-ctrl}{5}).
From the first line of output, it can be seen that the argument is still \inltex|\vala|, which means the character-wise operation in \inltex|\cmda| will not work when a control sequence is used as its input.
\begin{latexsample}[examplelabel=ex:trad-expan-ctrl,exampletitle={Expansion Control},noexport]
\def\cmda#1{%
  % do some character-wise operation with #1
  arg: \detokenize{#1} % output #1
}
\def\vala{val-a}
\def\valb{\vala}
\par\cmda{\vala}
\par\expandafter\expandafter\expandafter\cmda\expandafter{\vala}
\par\expandafter\expandafter\expandafter\cmda\expandafter{\valb}
\end{latexsample}
\noindent In order for function calls to work properly in \LT{}, it is the user's responsibility to preprocess the function arguments so that they are in the expected form of the function.
This procedure is known as expansion control.
One of the ways to conduct expansion control is to use the \TeX{} primitive \inltex|\expandafter|, whose usage is so complicated that it deserves its own tutorial \cite{bechtolsheim1988tutorial}.
On \lref{ex:trad-expan-ctrl}{8}, we demonstrate how to use \inltex|\expandafter| so that \inltex|\cmda| can correctly access the characters stored in \inltex|\vala|.
This approach does not work for all scenarios: notice how it fails when we use \inltex|\valb| as the input.

\LTT{} has provided user friendly interfaces to help \LT{} programmers from various backgrounds cope with the differences of \LT{}.
For more low-level details about \TeX{}/\LT{}, one is referred to \cite{knuth1984texbook,berry2017latex}.

% this section is placed in a separate file because it breaks Overleaf's syntax highlighting...

\subsection{Compiling Examples}

%This tutorial is based on examples.
To compile the examples in this tutorial, the minimum preamble required is:
\begin{latexsample}[examplelabel=ex:min-preamble,noexec]
\documentclass{article}
\usepackage{tikz} % load TikZ for some TikZ examples
\usepackage{expl3} % load latex3 packages
\end{latexsample}
\noindent The example code should be placed between \inltex|\begin{document}| and \inltex|\end{document}|. 
%\noindent The example code should be placed in the document body.
All examples have been tested with \TeX Live 2023.
For newer versions of \LaTeX{} compilers, there is no need to load the \verb|expl3| package explicitly (i.e., \lref{ex:min-preamble}{3} is optional).
% \footnotetext{Every listing has a unique index, which is shown at the bottom right corner. A line in the listing is referenced by <listing index>:<line number>. A range of lines in the listing are referenced by <listing index>:<line number 1>-<line number 2>.}
The source code of this tutorial and examples can be obtained from \url{https://github.com/xziyue/latex3-tutorial-latex-source}.

\section{\LTT{} Naming Conventions}

As mentioned in \cref{sec:important-concepts}, \LT{} is a macro language.
Macros can be used as variables to store data.
They can also be used as functions to apply operations on their arguments.
The duality of macros can make \LT{} source code very puzzling to read, because it is difficult to determine if a control sequence in the source code is a variable or a function.

To improve readability, \LT{} introduces a special naming system for macros so that functions and variables are clearly distinguishable based on names. 
The naming system also encodes additional information in the macros, such as the number of function arguments, the type of variables, and the scope of variables. 


\subsection{\LTT{} Mode}


To program using \LTT{}, one must first enter the \LTT{} mode.
In \LTT{} mode, the category codes of characters are changed for better programming experience.
The difference between \LTT{} mode and default \LT{} mode are summarized as follows:
\begin{enumerate}
    \item The category codes of \verb|_| and \verb|:| are set to 11 (letter), which allows them to be used in command names.
    \item The category codes of white space characters and newline characters are set to 9 (ignored), which causes \LTT{} code to ignore all spaces and line breaks.
\end{enumerate}
The command \inltex|\ExplSyntaxOn| is used to enter \LTT{} mode; the command \inltex|\ExplSyntaxOff| is used to exit \LTT{} mode.

In \LTT{} mode, we cannot use \verb|_| to denote subscript anymore because its category code has been changed to 11. Instead, a constant \inltex|\c_math_subscript_token| is defined in \LTT{} for this purpose.
There are several ways to typeset the ignored white space and newline characters in \LTT{} mode.
White spaces can be entered using \verb|~|, \verb|\ | (escaping space), or \verb|\space| depending on the scenario.
New lines can be triggered using the \inltex|\par| command.


\subsection{Names of Variables}\label{sec:name-of-variables}

In \LTT{}, the naming of variables and functions are different so that they are easily distinguishable. 
A \LTT{} variable can be public or private depending on semantics: public variables can be accessed by package users and private variables can only be accessed by package authors.
The name of each \LTT{} variable consists of four parts: \texttt{scope}, \texttt{module}, \texttt{description}, and \texttt{type}. 
We will describe the meaning of each part later.


\vspace*{0.5\baselineskip}
\par\noindent The name of a public variable should follow:
\begin{center}
\texttt{\string\<scope>\_<module>\_<description>\_<type>}
\end{center}
The name of a private variable should follow:
\begin{center}
\texttt{\string\<scope>\_\_<module>\_<description>\_<type>}
\end{center}
Private variable names have an extra underscore between \texttt{scope} and \texttt{module}.

\vspace*{0.5\baselineskip}
\par\noindent The detail of each part is described as follows.
\begin{itemize}
\item \texttt{scope}: a single letter identifying the scope of the variable
\begin{itemize}
\item \texttt{l}: local variable (i.e., to be accessed within current group)
\item \texttt{g}: global variable (i.e., to be accessed beyond current group)
\item \texttt{c}: constant
\end{itemize}
\item \texttt{module}: the name of the module where the variable is defined
\item \texttt{description}: the description of the variable
\item \texttt{type}: the variable type; some common types are shown in \cref{tbl:var-type}.
\end{itemize}
The \texttt{description} part can contain multiple segments separated by underscores. 
Some valid variables names are shown below.
\begin{itemize}
\item \inltex|\g_doc_variable_int|
\item \inltex|\l_doc_bg_color_r_fp|
\item \inltex|\l_doc_bg_color_g_fp|
\item \inltex|\l__mypkg_tmpa_seq|
\item \inltex|\c_left_brace_str|
\end{itemize}


\begin{table}[htpb]
\centering
\begin{tabular}{>{\ttfamily\centering}p{0.15\linewidth}p{0.7\linewidth}}
\toprule
\multicolumn{1}{c}{Type} & Description\\ \midrule
clist & comma separated list\\
dim & dimension\\
fp & floating point number\\
int & integer\\
seq & sequence\\
str & string\\
tl & token list\\
bool & boolean\\
regex & regular expression\\
prop & property list\\
ior & IO read\\
iow & IO write\\ \bottomrule
\end{tabular}
%TODO: insert reference
\caption{Commonly used variable types and their descriptions. Typically every \LTT{} module introduces one or more types.}
\label{tbl:var-type}
\end{table}

\subsection{Names of Functions}

To increase the readability, a function name in \LTT{} includes additional information about the arguments that it absorbs. 

\vspace*{0.5\baselineskip}
\par\noindent
The name of a public \LTT{} function should follow:
\begin{center}
\texttt{\string\<module>\_<description>:<arg-spec>}
\end{center}
The name of a private \LTT{} function should follow:
\begin{center}
\texttt{\string\_\_<module>\_<description>:<arg-spec>}
\end{center}
The semicolon is only used in \LTT{} function names, not in variable names.

\vspace*{0.5\baselineskip}
\par\noindent
The meaning of each part is described as follows.
\begin{itemize}
\item \texttt{module}: the name of the module where the function is defined
\item \texttt{description}: the description of the function
\item \texttt{arg-spec}: the argument specifications--it is made up of zero or more Latin alphabets describing the type of every function argument; each letter specifies the type of an argument (see \cref{tbl:arg-spec-type})
\end{itemize}
Some valid function names are shown below.
\begin{itemize}
\item \inltex|\group_begin:|
\item \inltex|\tl_put_right:Nn|
\item \inltex|\tl_gput_right:Nx|
\item \inltex|\__doc_do_something:Nnxx|
\end{itemize}

\begin{table}[htpb]
\centering
\small
\setlength\extrarowheight{3pt}
\begin{tabular}{>{\ttfamily\centering}cm{0.7\linewidth}}
\toprule
\multicolumn{1}{c}{\makecell{\texttt{arg-spec}\\ Letter}} & Description\\ \midrule
n & A token list argument\\
N & A single token argument\\
V & An argument passed by value\\
o & A token list argument expanded once\\
x & A token list argument expanded recursively\\
T & A true branch in conditional statements\\
F & A false branch in conditional statements\\
p & A parameter list\\
c & A token list argument that is fully expanded and treated as a command name\\
w & Weird arguments (the ones that do not fall in other categories)\\
\bottomrule
\end{tabular}
%TODO: insert reference and delete Section
\caption{Commonly used argument specification letters and their descriptions. More details are described in Section \textcolor{red}{???}.}
\label{tbl:arg-spec-type}
\end{table}

%TODO: think about how to deal with token and token lists

\subsection{The Use of \LTT{} Naming Conventions}

The \LTT{} naming conventions are voluntary. 
In general, \LTT{} does not check if the code follows \LTT{} naming conventions.
However, it is recommended that every \LTT{} programmer strictly follow these conventions to improve the readability and compatibility their code.
%TODO: insert reference and delete Section
In Section \textcolor{red}{???}, it can also be seen that \LTT{}'s expansion control mechanism sometimes requires functions to be named after \LTT{} conventions.


\section{Understanding \LTT{} Documentation (I.1.2)}

Currently, most information about \LTT{} is in \cite{l3interface}. 
It is essentially an itemized documentation on \LTT{} functions, scratch variables, and constants. 
We briefly describe how these items are presented in \cite{l3interface}.
\begin{itemize}
\item \LTT{} functions: the majority of the contents in \cite{l3interface} are about \LTT{} functions.
In \cite{l3interface}, each function is documented using a block similar to the one shown in  \cref{fig:l3-doc-func-example}.
The colored background is added in the figure to help understand the block.
The yellow part in \cref{fig:l3-doc-func-example} indicates the basic forms and the variants of the function. 
The first and third line in the yellow part are the basic forms for local and global assignment, respectively.
The second line shows the function variants for local assignment; the fourth line shows the function variants for global assignment.
%TODO: insert reference and delete Section
More details about function variants will be introduced in Section \textcolor{red}{???}.
The cyan part in \cref{fig:l3-doc-func-example} shows the basic usage of the function, where the meaning of each function argument is shown.
The magenta part in \cref{fig:l3-doc-func-example} shows the detailed description of the function.

\item Scratch variables: variables in \LTT{} need to be defined before use. 
\LTT{} has predefined a set of empty variables for convenience, which are known as scratch variables.
In \LTT{}, each module will usually define several scratch variables.
They are documented in specific sections, similar to the one demonstrated in \cref{fig:l3-scratch-var-example}.
The left hand side of \cref{fig:l3-scratch-var-example} shows the predefined scratch variables; the right hand side shows the description of the variables.
It is recommended to not use scratch variables (especially in large projects or generic packages), which can reduce the chances of variable collision.

\item Constants: \LTT{} defines a large number of constants such as the value of $\pi$, the value of $e$, special characters, etc. 
They are documented in specific sections, similar to the one demonstrated in \cref{fig:l3-constant-example}.
The left hand side of \cref{fig:l3-constant-example} shows the command name of the constant; the right hand side shows the description of the constant.

\end{itemize}

\begin{figure*}[tb]
\centering
\mdfdefinestyle{plainmdbox}{topline=false,leftline=false,rightline=false,bottomline=false}
\begin{minipage}[t]{0.52\linewidth}
\begin{mdframed}[backgroundcolor=yellow!50, style=plainmdbox]
\begin{tabular}{l}
\toprule
\inlpl|\tl_set:Nn|\\
{\color{gray}\verb|\tl_set|}\inlpl!:(NV|Nv|No|Nf|Nx|cn|cV|cv|co|cf|cx)! \\
\inlpl|\tl_gset:Nn| \\
{\color{gray}\verb|\tl_gset|}\inlpl!:(NV|Nv|No|Nf|Nx|cn|cV|cv|co|cf|cx)! \\ \bottomrule
\end{tabular}
\end{mdframed}
\end{minipage}
\begin{minipage}{0.4\linewidth}
\vspace*{3em}
\begin{mdframed}[backgroundcolor=cyan!50, style=plainmdbox]
\inlpl|\tl_set:Nn| {\ttfamily<\textsl{tl var}> \{<\textsl{tokens}>\}}
\end{mdframed}
\end{minipage}\\
\begin{minipage}[t]{0.15\linewidth}
\phantom{}
\end{minipage}
\begin{minipage}{0.82\linewidth}
\begin{mdframed}[backgroundcolor=magenta!50, style=plainmdbox]
Sets <\textit{tl var}> to contain <\textit{tokens}>, removing any previous content from the variable.
\end{mdframed}
\end{minipage}
\caption{An example of function documentation excerpted from \LTT{} documentation. 
Colored background is added in this figure to help understand the documentation.}
\label{fig:l3-doc-func-example}
\end{figure*}

\begin{figure*}[tb]
\centering
\begin{minipage}[t]{0.15\linewidth}
\begin{tabular}{l}
\toprule
\inlpl|\l_tmpa_tl|\\
\inlpl|\l_tmpb_tl| \\ \bottomrule
\end{tabular}
\end{minipage}
\begin{minipage}[t]{0.83\linewidth}
\vspace*{-1em}
Scratch token lists for local assignment.
These are never used by the kernel code, and so are safe for use with any \liii--defined function.
However, they may be overwritten by other non-kernel code and so should only be used for short-term storage.
\end{minipage}
\caption{An example of scratch variables excerpted from \LTT{} documentation.}
\label{fig:l3-scratch-var-example}
\end{figure*}

\begin{figure*}[tb]
\centering
\begin{minipage}[t]{0.15\linewidth}
\begin{tabular}{l}
\toprule
\inlpl|\c_pi_fp|\\ \bottomrule
\end{tabular}
\end{minipage}
\begin{minipage}[t]{0.83\linewidth}
The value of $\pi$. This can be input directly in a floating point expression as \texttt{pi}.
\end{minipage}
\caption{An example of scratch variables excerpted from \LTT{} documentation.}
\label{fig:l3-constant-example}
\end{figure*}


\section{Variables and Functions}

In this section, we introduce the definition and usage of variables and functions in \LTT{}.

\noindent \textbf{Variables.}
Each \LTT{} type is associated with a set of dedicated functions for defining, modifying, and accessing variables.
Most \LTT{} types provide the following functions for managing variables:
\begin{itemize}
    \item \verb|new| functions to define a new variable (e.g., \inltex|\tl_new:N|, \inltex|\int_new:N|)
    \item \verb|set| functions to set a variable locally (e.g., \inltex|\tl_set:Nn|, \inltex|\int_set:Nn|)
    \item \verb|gset| functions to set a variable globally (e.g., \inltex|\tl_gset:Nn|, \inltex|\int_gset:Nn|)
    \item \verb|item| functions to extract an element from data containers by index or key (e.g., \inltex|\tl_item:Nn|, \inltex|\seq_item:Nn|, \inltex|\prop_item:Nn|)
    \item \verb|get| functions to get a specific element from data containers by index (e.g., \inltex|\seq_get_left:N|, \inltex|\seq_get_right:N|)
    \item \verb|use| functions to put the value of a variable into the document (e.g., \inltex|\tl_use:N|, \inltex|\seq_use:Nn|)
    \item \verb|show| functions to output the value of a variable to the terminal for debug purposes (e.g., \inltex|\tl_show:N|, \inltex|\seq_show:N|)
\end{itemize}
Note that not all \LTT{} modules follow this convention. In general, to use a variable in \LTT{}, one should follow these steps:
\begin{enumerate}
    \item Determine the correct variable type and call the corresponding definition function
    \item Determine the scope, visibility (private/public), and description of the variable and name it using \LTT{} naming conventions
    \item Use functions in the variable's module to operate on the variable
\end{enumerate}


\begin{latexsample}[examplelabel={ex:var-example},exampletitle={Using Variables}]
\ExplSyntaxOn
% define a new integer varaible
\int_new:N \l_my_integer_int
% set the integer value to be the result of an integer expression
\int_set:Nn \l_my_integer_int {2*(3+1)}
% output the integer value
The~integer~is~\int_use:N \l_my_integer_int
\ExplSyntaxOff
\end{latexsample}

\noindent\textbf{Functions.}
In \LTT{}, functions are defined using the \inltex|\cs_new:Npn| function.
Although related functions such as \inltex|\cs_new_protected:Npn|, \inltex|\cs_new_nopar:Npn|, and \inltex|\cs_new_protected_nopar:Npn| exist, we only use \inltex|\cs_new:Npn| in this tutorial since it should be used in most scenarios. 
Readers are encouraged to read the \LTT{} documentation for more information about the related functions.
In general, to define a function, one should follow these steps:
\begin{enumerate}
    \item Determine the number of function arguments and the argument specification of each argument; in most cases, the argument specification is either \verb|N| or \verb|n|
    \item Determine the scope, visibility (private/public), and description of the function and name it using \LTT{} naming conventions
\end{enumerate}
\LTT{} also provides useful utility functions such as \inltex|\cs_set_eq:NN| for copying the definition of functions, and \inltex|\cs_meaning:N|, \inltex|\cs_show:N| for displaying the definition of functions.

\begin{latexsample}[examplelabel={ex:func-example},exampletitle={Using Functions}]
\ExplSyntaxOn
% define a new function that accepts two n-type arguments
\cs_new:Npn \l_my_func:nn #1#2
{
    [arg1=#1][arg2=#2]
}
% call the function with two arguments
\par\l_my_func:nn {first} {second}
% show the definition of the function
\par\cs_meaning:N \l_my_func:nn
% copy the definition of the new function
\cs_set_eq:NN \l_my_func_copy:nn \l_my_func:nn
% show the definition of the copied function
\par\cs_meaning:N \l_my_func_copy:nn
\ExplSyntaxOff
\end{latexsample}

\bibliographystyle{tugboat}
\bibliography{main.bib}

\makesignature

\end{document}