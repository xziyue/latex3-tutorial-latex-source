% !BIB TS-program = bibtex
% !TeX TS-program = xelatex
\documentclass{ltugboat}

\usepackage[T1]{fontenc}
\usepackage{lmodern}
\usepackage{amsmath, amssymb}

\DeclareFontFamily{T1}{DejaVuSansMono-TLF}{\hyphenchar\font-1}
\DeclareFontShape{T1}{DejaVuSansMono-TLF}{m}{n}{<-> s*[0.75]DejaVuSansMono-tlf-t1}{}
\DeclareFontShape{T1}{DejaVuSansMono-TLF}{m}{it}{<-> s*[0.75]DejaVuSansMono-Oblique-tlf-t1}{}
\DeclareFontShape{T1}{DejaVuSansMono-TLF}{b}{n}{<-> s*[0.75]DejaVuSansMono-Bold-tlf-t1}{}
\DeclareFontShape{T1}{DejaVuSansMono-TLF}{b}{it}{<-> s*[0.75]DejaVuSansMono-BoldOblique-tlf-t1}{}

\DeclareFontFamily{T1}{lmtt}{\hyphenchar\font-1}
\DeclareFontShape{T1}{lmtt}{m}{n}{<-> s*[0.9]ec-lmtt10}{}
\DeclareFontShape{T1}{lmtt}{m}{it}{<-> s*[0.9]ec-lmtti10}{}
\DeclareFontShape{T1}{lmtt}{m}{sl}{<-> s*[0.9]ec-lmtto10}{}
\usepackage[english]{babel}
\usepackage{metalogo}
\usepackage{tcolorbox}
\usepackage{booktabs}
\usepackage{microtype}
\usepackage{expl3}
\usepackage{mdframed}
\usepackage{chngcntr}
\usepackage{etoolbox}
\usepackage{adjustbox}
\usepackage{supertabular}
\usepackage{makecell}
\usepackage{tikz}
\usepackage{ragged2e}
\usepackage{float}
\usepackage{array}
\usepackage{xcolor}

\ExplSyntaxOn
\file_if_exist:nTF{USE_SHELL_ESCAPE}
{
    \file_if_exist:nTF{MINTED_FINALIZE}
    {
        \iow_term:n {=====~Document~is~compiled~in~finalization~mode~=====}
        \usepackage[finalizecache]{minted}
    }
    {
        \iow_term:n {=====~Document~is~compiled~in~development~mode~=====}
        \usepackage{minted}
    }
}
{
    \iow_term:n {=====~Document~is~compiled~in~submission~mode~=====}
    \usepackage[frozencache]{minted}
}
\ExplSyntaxOff


%\usepackage[breaklinks,colorlinks,pdfa]{hyperref}
\usepackage[breaklinks]{hyperref}
\usepackage[capitalise]{cleveref}


\tcbuselibrary{listings, minted, skins, breakable, xparse, hooks}

% a quick command for LaTeX3
\newcommand*{\liii}{\LaTeX3}

\renewcommand{\MintedPygmentize}{./my_pygmentize}
% other code highliting setup
\newmintinline[inltex]{tex_lexer.py:Tex3Lexer}{breaklines, breakanywhere, fontfamily=DejaVuSansMono-TLF}
\newmintinline[inlpy]{python}{fontfamily=DejaVuSansMono-TLF}
\newmintinline[inlpl]{text}{}

% set the style of the list of examples
% https://tex.stackexchange.com/questions/86711/tcolorbox-list-of-listings
\newcommand{\ListOfCodeExampleName}{List of Examples}
\makeatletter
\newcommand{\ListOfCodeExample}{\section*{\ListOfCodeExampleName}\@starttoc{CodeExample}}
\makeatother

% the counter for numbered code examples
\newcounter{codeexample}
\counterwithin{codeexample}{section}

%%%%%%%%%%%%%%%%%%%%%%%%%%%%%%%%%%%%%%%%%%%%%%%%%%%%%%%%%%%%%%%%%%
% this segment of code is to implement arbitrary code
% cross-referencing within the article
%%%%%%%%%%%%%%%%%%%%%%%%%%%%%%%%%%%%%%%%%%%%%%%%%%%%%%%%%%%%%%%%%%
\makeatletter
\ExplSyntaxOn
\bool_new:N \g_lst_line_ref_bool
\tl_new:N \g_lst_label_tl
\tl_new:N \l_lst_tmpa_tl
\tl_new:N \l_lst_tmpb_tl
\prop_new:N \g_lst_index_prop

\cs_set:Npn \__lst_write_aux:n #1 {
    \immediate\write\@auxout{#1}
}

\newcommand{\EnableLstRef}{\bool_gset_true:N \g_lst_line_ref_bool}
\newcommand{\DisableLstRef}{\bool_gset_false:N \g_lst_line_ref_bool}
\newcommand{\SetLstRefLabel}[1]{\tl_gset:Nn \g_lst_label_tl {#1}}
\newcommand{\LstPutLookup}[1]{
    \tl_set:Nn \l_lst_tmpb_tl {
       \string\expandafter\string\gdef\string\csname\space @@lst-lookup-#1\string\endcsname{\TheLstCounter}
    }
    \exp_args:Nx \__lst_write_aux:n \l_lst_tmpb_tl
}
\newcommand{\LstUseLookup}[1]{
    \cs_if_exist:cTF {@@lst-lookup-#1} {
        \use:c {@@lst-lookup-#1}
    } {
        ?
    }
}


\cs_set:Npn \__lst_render_line: {
    \textcolor[rgb]{0.5,0.5,0.5}{
    \ttfamily\scriptsize
    \arabic{FancyVerbLine}
    }
}

\cs_set:Npn \__lst_make_hypertarget:nn #1 #2{
   \raisebox{1em}{\hypertarget{#1}{}}#2 
}

\renewcommand{\theFancyVerbLine}{
    \bool_if:NTF \g_lst_line_ref_bool {
        \tl_set:Nx \l_lst_tmpa_tl {\arabic{FancyVerbLine}}
        \exp_args:Nx \__lst_make_hypertarget:nn {\g_lst_label_tl-\l_lst_tmpa_tl}{
            \__lst_render_line:
        }
    }{
        \__lst_render_line:
    }

}

\int_new:N \g_lst_counter_int
\newcommand{\AddLstCounter}{\int_gincr:N \g_lst_counter_int}
\newcommand{\TheLstCounter}{\int_use:N \g_lst_counter_int}

% user command to cross reference a line in the lising
% #1: listing name
% #2: line number
% #3 (optional): would output a range of lines from #2 to #3 if present
\DeclareDocumentCommand{\lref}{mmo}{
    \IfValueTF{#3}{Lines}{Line}~\hyperlink{#1-#2}{
        \LstUseLookup{#1}:
        \IfValueTF{#3}{#2-#3}{#2}
    }
}

% user command to cross referenec a listing
\DeclareDocumentCommand{\lstref}{m}{
    Listing~\hyperlink{#1-1}{
        \LstUseLookup{#1}
    }
}

\ExplSyntaxOff
\makeatother
%%%%%%%%%%%%%%%%%%%%%%%%%%%%%%%%%%%%%%%%%%%%%%%%%%%%%%%%%%%%%%%%%%
%%%%%%%%%%%%%%%%%%%%%%%%%%%%%%%%%%%%%%%%%%%%%%%%%%%%%%%%%%%%%%%%%%

% basic style for examples
\tcbset{
  codesample/.style={
    enhanced,
    breakable,
    beforeafter skip=3ex,
    listing engine=minted,
    minted language=tex_lexer.py:Tex3Lexer,
    fontlower=\sffamily\small,
    minted options={
      fontsize=\fontsize{9}{9},
      autogobble,
      breaklines,
      obeytabs,
      tabsize=2,
      linenos,
      numbersep=2mm,
      xleftmargin=4mm,
      fontfamily=DejaVuSansMono-TLF
    },
    fonttitle=\normalfont,
    colback=white,
    colframe=black!60,
    boxrule=0.8pt,
    left=1mm,
    right=1mm,
    top=0.5mm,
    bottom=0.5mm,
    before upper pre={\AddLstCounter},
    overlay unbroken and last={
        \node[font=\sffamily\bfseries\scriptsize, anchor=south east] at (frame.south east) {\color[rgb]{0.5,0.5,0.5} \TheLstCounter};
    }
  }
}

%%%%%%%%%%%%%%%%%%%%%%%%%%%%%%%%%%%%%%%%%%%%%%%%%%%%%%%%%%%%%%%%%%
% this segment of code is to implement code exporting
% if \ExportExamplestrue is called, then the source code of each
% example will be output to /examples
%%%%%%%%%%%%%%%%%%%%%%%%%%%%%%%%%%%%%%%%%%%%%%%%%%%%%%%%%%%%%%%%%%
\newif\ifExportExamples

\ifExportExamples

% define our custom VerbatimOut like environment for latexsample
\makeatletter

\ExplSyntaxOn

\cs_set:Npn \__doc_write:Nn #1#2 {
    \immediate\write#1{#2}
}
\cs_generate_variant:Nn \__doc_write:Nn {Nx}
\cs_set_eq:NN \MyWriteA \__doc_write:Nn
\cs_set_eq:NN \MyWriteB \__doc_write:Nx
\cs_set_eq:NN \PCTSGN \c_percent_str

\cs_set:Npn \__doc_open_out:Nn #1#2 {
    \immediate\openout#1 #2\relax
}
\cs_generate_variant:Nn \__doc_open_out:Nn {Nx}
\cs_set_eq:NN \MyOpenOut \__doc_open_out:Nx

\tl_new:N \l_doc_tmpa_tl
\tl_new:N \ExampleFilename

\cs_set:Npn \UpdateExampleFilename {
    \tl_set_eq:NN \l_doc_tmpa_tl \thecodeexample
    \regex_replace_all:nnN {\.} {-} \l_doc_tmpa_tl
    \tl_set_eq:NN \ExampleFilename \l_doc_tmpa_tl
}

\ExplSyntaxOff

\newwrite\FV@OutFileA
\newwrite\FV@OutFileB

\def\MyProcessLine#1{%
    \immediate\write\FV@OutFileA{#1}%
    \immediate\write\FV@OutFileB{#1}%
}

\def\latexsample{\FV@Environment{}{latexsample}}
\def\FVB@latexsample#1{%
  \@bsphack
  \begingroup
    \FV@UseKeyValues
    \FV@DefineWhiteSpace
    \def\FV@Space{\space}%
    \FV@DefineTabOut
    \let\FV@ProcessLine\MyProcessLine
    % increment example counter
    \refstepcounter{codeexample}
    \UpdateExampleFilename
    %\def\FV@ProcessLine{\immediate\write\FV@OutFile}%
    % copy1: this is to be read back and compiled in the document
    \MyOpenOut\FV@OutFileA{temp-example.vrb}
    %\immediate\openout\FV@OutFileA temp-example.vrb\relax
    % copy2: this is to be saved in examples folder
    %\immediate\openout\FV@OutFileB temp-example2.vrb\relax
    \MyOpenOut\FV@OutFileB{examples/example-\ExampleFilename.tex}
    % write the comment in the example
    \MyWriteB\FV@OutFileB{\PCTSGN\PCTSGN\space Example \thecodeexample: #1}
    % save #1
    \gdef\TempExampleTitle{#1}
    \let\FV@FontScanPrep\relax
%% DG/SR modification begin - May. 18, 1998 (to avoid problems with ligatures)
    \let\@noligs\relax
%% DG/SR modification end
    \FV@Scan}
\def\FVE@latexsample{%
\immediate\closeout\FV@OutFileA%
\immediate\closeout\FV@OutFileB%
\endgroup\@esphack%
% input listing
\inputlatexsample{\TempExampleTitle}{temp-example.vrb}
}

\DefineVerbatimEnvironment{latexsample}{latexsample}{}

\makeatother


\newtcbinputlisting{\inputlatexsample}[2]{
  codesample,
  title=\GenExampleTitle{#1},
  listing file={#2}
}

\newcommand{\GenExampleTitle}[1]{%
    %\refstepcounter{codeexample}
    \hspace*{0.5em}
    Example \thecodeexample:~#1
    \addcontentsline{CodeExample}{subsection}{\protect\numberline{\thecodeexample}#1}
}

%TODO: fix the environment when exporting!

\else

\newcommand{\GenExampleTitle}[2]{%
    \refstepcounter{codeexample}
    \IfValueT{#2}{\label{#2}}
    \hspace*{0.5em}
    Example \thecodeexample:~#1
    \addcontentsline{CodeExample}{subsection}{\protect\numberline{\thecodeexample}#1}
}

\DeclareTCBListing{latexsample}{m!o}{
    codesample,
    title=\GenExampleTitle{#1}{#2},
    IfValueTF={#2}
    {
        before upper app={\EnableLstRef\SetLstRefLabel{#2}\LstPutLookup{#2}},
        after app={\DisableLstRef}
    }
    {}
}

%\newtcblisting{latexsample}[1]{
%  codesample,
%  title=\GenExampleTitle{#1}
%}

\fi


%%%%%%%%%%%%%%%%%%%%%%%%%%%%%%%%%%%%%%%%%%%%%%%%%%%%%%%%%%%%%%%%%%
%%%%%%%%%%%%%%%%%%%%%%%%%%%%%%%%%%%%%%%%%%%%%%%%%%%%%%%%%%%%%%%%%%



% listing only example
\DeclareTCBListing{latexsample**}{!o}{
    codesample,
    listing only,
    IfValueTF={#1}
    {
        before upper app={\EnableLstRef\SetLstRefLabel{#1}\LstPutLookup{#1}},
        after app={\DisableLstRef}
    }
    {}
}

% anonymous example
\DeclareTCBListing{latexsample*}{!o}{
    codesample,
    IfValueTF={#1}
    {
        before upper app={\EnableLstRef\SetLstRefLabel{#1}\LstPutLookup{#1}},
        after app={\DisableLstRef}
    }
    {}
}

%\newtcblisting{latexsample*}{
%  codesample
%}

% listing of any language
\newtcblisting{codesample}[1]{
  codesample,
  listing only,
  minted language=#1
}



\makeatletter
% better URL line breaking
\g@addto@macro{\UrlBreaks}{\UrlOrds}

%\newcommand{\inlcolorbox}[2]{
%    \bgroup
%    %\setlength{\fboxsep}{0pt}
%    \colorbox{blue!30}{#2}
%    \egroup
%}

%\patchcmd{\minted@inputpyg@inline}{\colorbox}{\inlcolorbox}{}{\GenericError{}{cannot patch minted}{}{}}
%\let\old@minted@inputpyg@inline\minted@inputpyg@inline
%\renewcommand{cmd}{def}

\crefname{codeexample}{Example}{Examples}
\newcommand{\exfullref}[1]{Example \ref{#1} (\lstref{#1})}

\makeatother

\title{Modern \LaTeX~programming: an example based \liii~tutorial}
\author{Ziyue Xiang}
\address{Purdue University}
\netaddress{ziyue.alan.xiang (at) gmail (dot) com}
\personalURL{https://www.alanshawn.com}
\ORCID{0000-0001-6054-5801}

\begin{document}

\begin{abstract}

\end{abstract}

\maketitle

\tableofcontents
\ListOfCodeExample


\section{Introduction}

There is no doubt that \LaTeX~is viewed as a typesetting language by most of the users.
The programming aspect of \LaTeX~is often overlooked by many.
In practice, many large and structured documents can benefit from the programming capabilities provided by \LaTeX. 
Even understanding the most basic programming principles in \LaTeX~can greatly facilitate the efficiency of generating figures or tables that are made up of similar and repeated sub-structures.
The infrastructure provided by \LaTeXe\ \cite{berry2017latex} is already Turing complete, which means its programming capabilities are identical to those of Python \cite{vanrossum2010python} and C \cite{ritchie1988c}.
However, the syntax and conventions of \LaTeXe~are nonstandardized and obsolete compared to the mainstream programming languages now.
This makes learning \LaTeXe~programming more difficult for today's \LaTeX~users.

%Many people view \LaTeX\ as a typesetting language and overlook the importance of programming in document generation process. 
%In reality, many large and structural documents can benefit from
%a programming backend, which enhances layout standardization, symbol
%coherence, editing speed and many other aspects. Despite the fact the
%standard \LaTeX\ (\LaTeXe) is already Turing complete, which means
%it is capable of solving any programming task, the design of numerous
%programming interfaces is highly inconsistent due to the long history
%of \LaTeX. This makes programming with \LaTeXe\ extremely daunting, 
%even for seasoned computer programmers.

In order to modernize the programming interfaces in \LaTeX, the \liii~programming interfaces are introduced \cite{mittelbach2020quo}.
Unfortunately, the learning materials for this tool chain is scarce.
One of the few resources available for new learners is The \liii\ Interfaces \cite{l3interface}, which is an API documentation that is difficult to for one to start with.
Therefore, in this material we intend to provide an example based \liii~tutorial for \liii~learners with sufficient background in computer programming.
That is, the reader is expected to understand basic structures (e.g., loops, conditional branches) as well as data types (e.g., integers, floating point numbers, strings) in programs. 
It would be helpful if the reader also understands the basic principles of the C programming language \cite{ritchie1988c}.

%To make programming in \LaTeX\ easier, the \liii\ programming interface
%is introduced, which aims to provide modern-programming-language-like
%syntax and library for \LaTeX\ programmers. Unfortunately, learning
%materials for this wonderful language is scarce. 
%which is essentially an API documentation that is not designed for 
%introductory purposes. This situation may have barred many \LaTeX\ 
%users from utilizing the generic programming capabilities of \LaTeX.
%Therefore, this article intends to provide an easy-to-understand 
%tutorial for \LaTeX\ users with computer programming knowledge.
%Hopefully, readers can improve their \LaTeX\ editing efficiency and 
%document quality after understanding \liii.


Since the \liii~project has accumulated a huge code base over the years, it is infeasible to cover all of its functionalities in one tutorial.
In this tutorial, we focus on the most frequently used components in \liii. 
The complete API documentation can be found in \cite{l3interface}. 
In the upcoming section titles, if parentheses are present, then the section number in \cite{l3interface} corresponding to the section in this tutorial is shown in the parentheses.

\subsection{Motivations of \liii}

As mentioned above, \LaTeXe~is already Turing complete and serves as the building blocks of many existing packages. In this section, we describe the problems of traditional \LaTeX~programming, which justifies the reason why \liii~is developed despite already having the powerful \LaTeXe~and many other existing packages.

\paragraph{Nonuniform interface} 

Outside \liii, the interfaces provided by traditional \LaTeX~are not standardized. They suffer from the following disadvantages.

% 1: function/variable naming
% 2: overlapping/scattered packages
% 3: numerical computations

Firstly, the mechanism of \LaTeX~can affect the readability of traditional \LaTeX~code. 
In \LaTeX, functions and variables are all declared as control sequences (see Chapter 3 of \cite{knuth1984texbook}).
When a function is invoked, it is expected to execute a series of predefined procedures, whereas variables are used to store values only.
In \LaTeX, we can declare functions that absorbs one or multiple arguments. 
The homogeneous nature of functions and variables can make reading traditional \LaTeX\ source code difficult.
%Unlike many other programming languages that enclose the arguments with parentheses, \LaTeX~does not require delimiters between the function and its arguments. 
In the example below, we define 6 control sequences, where \inltex|\ta| and \inltex|\td| are functions, and the rest are variables.
\begin{latexsample*}[ex:func-var-def]
\newcommand{\ta}[2]{[arg1={#1}, arg2={#2}]}
\newcommand{\tb}{$\alpha$}
\newcommand{\tc}{$\beta$}
\newcommand{\td}[1]{[arg3={#1}]}
\newcommand{\te}{$\gamma$}
\newcommand{\tf}{$\delta$}
\ta\tb\tc\td\te\tf
\end{latexsample*}
\noindent On line \lref{ex:func-var-def}{7}\footnotemark, we call functions \inltex|\ta| and \inltex|\td| with their respective arguments (stored in variables), as well as output the value of \inltex|\tf|.
In appearance, line 7 is six control sequences placed next to each other.
It is very difficult to understand the code unless the programmer finds out which control sequences are functions and how many arguments each function uses.
This makes the source code of some \LaTeXe\ packages challenging to read.

\footnotetext{Every listing has a unique index, which is shown at the bottom right corner. A line in the listing is referenced by <listing index>:<line number>.}

Secondly, in traditional \LaTeX~the implementation of many fundamental programming capabilities are provided by external packages.
As a result, the functionalities of multiple packages may overlap. 
For example, to compare the equality of two strings, we can use \inltex|\ifthenelse| and \inltex|\equal| from \verb|ifthen| package \cite{pkg:ifthen};
we can use \inltex|\pdfstrcmp| from \verb|pdftexcmds| package \cite{pkg:pdftexcmds};
we can also use \inltex|\IfStrEq| from \verb|xstring| package \cite{pkg:xstring}. 
The use of multiple similar packages is likely to cause redundancy and compatibility issues.
The lack of a centralized documentation and comparison for these similar packages also increases the learning cost of traditional \LaTeX~programming.
%Therefore, \liii\ is to provide a set of unified and standardized interfaces for all possible \LaTeX\ variable types.

\paragraph{Expansion control} 

% two situations: when trying to store something in a variable
% when passing arguments to a function
% only concerns arguments

Unlike generic programming languages, \LaTeX\ does not have support for types.
Programming components such as variables, functions and function arguments are all treated in the same way as macros.
As a result, \LaTeX~programmers usually need to more precisely control how variables are defined and how functions are called. These techniques are known as expansion control.

Expansion control is mostly required in two  scenarios.

Fundamentally, \TeX\ works by doing: commands are substituted by their definition,
which is subsequently replaced by definition's definition, until 
something irreplaceable is reached (e.g. text or \TeX\ primitives). 
This process is called \emph{expansion}. The mechanism of expansion
may sound simple and straightforward. However, it usually requires
a lot of manual fine-tuning in practice.

Consider the example below. We know that the \inltex|\uppercase| macro
capitalize English letters, which renders the first output line in all 
caps. But if we store some text in \inltex|\myname| and then apply
\inltex|\uppercase| to the command, we can see that the output is
\emph{not} turned into uppercase letters.

\begin{latexsample*}[]
\par\uppercase{Alan Xiang}
\newcommand*{\myname}{Alan Xiang}
\par\uppercase{\myname}
\end{latexsample*}

Why would this happen? Let us dig into how \inltex|\uppercase| works. 
The \inltex|\uppercase| macro scans each token\footnotemark inside its argument 
group one by one. If an English letter is encountered, its uppercase 
form is left in the output stream. If a command is encountered, it 
will not try to apply \inltex|\uppercase| to the content of the command. 
Instead, the command itself will be placed into the output stream. 
In this case, \inltex|\myname| will be left untouched in the output, 
which is subsequently expanded to its original definition.

(More details about tokens can be found in and Chapter 7 of \cite{knuth1984texbook})

\footnotetext{Tokens are smallest units that \TeX\ compilers work 
with. For now, we can consider a token to be either a character 
or command. For more about \TeX\ tokens, see \cite{overleaf-token}.}

What if we also want to capitalize the content of \inltex|\myname| as 
well? To achieve this, we need to fine-tune the expansion process by 
changing the \emph{order} of expansion. That is, to expand 
\inltex|\myname| before \inltex|\uppercase|. In this way, the 
\inltex|\uppercase| command will receive the content of 
\inltex|\myname| in the form of English letters, which allows 
capitalization to function correctly.

In \LaTeX, the classic way of controlling the order of expansion is
via the \inltex|\expandafter| macro, which it is notoriously
difficult to use. According to \emph{A Tutorial on \cs{expandafter}} 
\cite{bechtolsheim1988tutorial}, to reverse the expansion of a series
of $n$ tokens, the $i$th token has to be preceded by $2^{n-i}-1$
\inltex|\expandafter|s. The exponential growth of the number of 
\inltex|\expandafter|s greatly reduces the readability of source code
and increases the chances of mistakes. 
For example, in Joseph Wright's answer to an expansion-related
question on \TeX\ StackExchange \cite{tex-se-expanding}, a total of 26 
\inltex|\expandafter|s are used to reorder the expansion of merely 4
arguments.  To avoid this annoyance, one of the key features
of \liii\ is to provide simple and reliable expansion control.



\paragraph{Modernized experience} \TeX\ was first designed in the late 1970s, when
computer hardware and programming languages were prototypes compared to
their contemporary counterparts. As a result, \TeX\ and \LaTeX\ contain
quirky usages that may seem odd for programmers today. For example, to
multiply a counter variable by 3, one writes 
\inltex|\multiply|\inltex|\counter| \inltex|by| \inltex|3|; to invoke the \verb|date| command via
the terminal, one writes \inltex|\immediate\write18{date}|. 
It can be seen that these syntaxes are either outdated or perplexing. 
In a fairly popular language nowadays 
(e.g. Python), these two tasks can be done by \inlpy|counter*=3|
and \inlpy|os.system('date')|, whose code possesses superior
simplicity and interpretability. \liii\ attempts to modernize 
the \LaTeX\ language by adapting to modern-language-like syntaxes
and introducing a naming system that makes \LaTeX\ code more
readable.

\subsection{Compiling Examples}

This tutorial is based on examples.
To compile the examples, the minimum preamble required is:
\begin{latexsample**}[ex:min-preamble]
\documentclass{article}
\usepackage{tikz} % load TikZ for some TikZ examples
\usepackage{expl3} % load latex3 packages
\end{latexsample**}
\noindent The example code should be placed between \inltex|\begin{document}| 
and \inltex|\end{document}|. 
All examples are tested with \TeX Live  2020 on Ubuntu 20.04.
For newer versions of \LaTeX~compilers, there is no need to load the \verb|expl3| package explicitly (i.e., line \lref{ex:min-preamble}{3} is optional).

The source code of this tutorial can be obtained from \url{https://github.com/xziyue/latex3-tutorial-latex-source}.


\section{\liii\ Naming Conventions (I-1)}

In Python or C++, if we see \inlpy|a(b);|, we can 
tell \inlpy|a| is a function and \inlpy|b| is its 
argument. However, in \LaTeX, if we see \inltex|\a\b|, 
there are be two possibilities:
\begin{itemize}
\item \inltex|\a| is a function and \inltex|\b| is its argument
\item Both \inltex|\a| and \inltex|\b| are variables
\end{itemize}
The syntactic design of \LaTeX\ makes it difficult
to distinguish between functions and variables, for
each control sequence can either be a function that 
receives arguments or a variable that absorbs nothing.
It can lead to confusion when one is trying to understand
others' source code. Therefore, \liii\ introduces a set
of naming rules that encode important information into
the name of control sequences as a way to improve 
readability.

Before discussing \liii\ naming conventions, let us take a diversion 
to look at the low-level design of \LaTeX\ and find out 
how we can use non-English characters in command names.

\subsection{Category Code \& Command Name}

When the \LaTeX\ compiler reads a source file, it will read
and process each character one by one. For each character in 
the file, in addition to its character code, \LaTeX\ compiler 
will also assign a \emph{category code} based on current
category code table. The default \LaTeX\ category code table
is shown in Table \ref{table:catcode-table}.

\begin{center}
\small
\tabletail{\hline}
\tablehead{
\hline
\makecell{Category\\Code} & Description & 
\makecell{Character(s)}\\ \hline
}
\bottomcaption{Default \LaTeX\ category code table \cite{overleaf-catcode}. 
Characters surround by single quotes indicate their C-style
representation.}
\label{table:catcode-table}
\begin{supertabular}{|c|>{\centering}m{0.4\linewidth}|c|}
0 & Escape character: tells \LaTeX\ 
to start looking for a command & \verb|\| \\  \hline
1 & Start of group & \verb|{|\\ \hline
2 & End of group & \verb|}|\\ \hline
3 & Toggle math mode & \verb|$|\\ \hline
4 & Alignment tab & \verb|&|\\ \hline
5 & End of line & \verb|'\r'|\\ \hline
6 & Macro parameter & \verb|#| \\ \hline
7 & Superscript & \verb|^| \\ \hline
8 & Subscript & \verb|_| \\ \hline
9 & Ignored character & \verb|'\0'| \\ \hline
10 & Spacer & \verb|'\32'|, \verb|'\t'|\\ \hline
11 & Letter & \verb|A|--\verb|Z|, \verb|a|--\verb|z|, \ldots\\ \hline
12 & Other & \verb|0|--\verb|9|, \verb|+|, \verb|@|\ldots \\ \hline
13 & Active character: used for single character commands & 
\verb|~|\ldots\\ \hline
14 & Comment character: ignore everything that follows until 
end of line & \verb|%|\\ \hline
15 & Invalid character: not allowed in \verb|.tex| files & 
\verb|'\127'|\ldots \\ \hline
\end{supertabular}
\end{center}

\LaTeX\ reacts to each character according to its category code
instead of character code. If we change the category code associated
with a character, we can completely change the \emph{meaning} of that
character. For example, if we assign category code 7 to \verb|_| and
category code 8 to \verb|^|, we can use \verb|_| to denote superscript
and \verb|^| to denote subscript.

\begin{latexsample}{Doing 123}
\ExplSyntaxOn
\tl_set:Nn \l_tmpa_tl {A}
\group_begin:
\tl_set:Nn \l_tmpa_tl {B}
\par value~inside~group:~\tl_use:N \l_tmpa_tl
\group_end:
\par value~outside~group:~\tl_use:N \l_tmpa_tl

\tl_set:Nn \l_tmpb_tl {A}
\group_begin:
\tl_gset:Nn \l_tmpb_tl {B}
\par value~inside~group:~\tl_use:N \l_tmpb_tl
\group_end:
\par value~outside~group:~\tl_use:N \l_tmpb_tl
\ExplSyntaxOff
\end{latexsample}


\bibliographystyle{tugboat}
\bibliography{main.bib}

\makesignature

\end{document}