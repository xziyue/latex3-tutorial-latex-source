\documentclass{ltugboat}
\usepackage[T1]{fontenc}
\usepackage{lmodern}
\usepackage[english]{babel}
\usepackage{metalogo}
\usepackage{tcolorbox}
\usepackage{microtype}
\usepackage{expl3}
\usepackage{etoolbox}
\usepackage{supertabular}
\usepackage{makecell}
\usepackage{tikz}
\usepackage{ragged2e}
\usepackage{float}
\usepackage{array}
\usepackage{xcolor}
%\usepackage[breaklinks,hidelinks,pdfa]{hyperref}
\usepackage{hyperref}

\tcbuselibrary{listings, minted, skins, breakable}

\newcommand*{\liii}{\LaTeX3}

% line number style
\renewcommand{\theFancyVerbLine}{
\ttfamily
\textcolor[rgb]{0.5,0.5,0.5}{\scriptsize
{\arabic{FancyVerbLine}}}}

\newcommand*{\liiilexer}{tex_lexer.py:Tex3Lexer -x}
\newmintinline[inltex]{\liiilexer}{}
\newmintinline[inlpy]{python}{}

\tcbset{
  codesample/.style={
    enhanced,
    breakable,
    listing engine=minted,
    minted language=\liiilexer,
    minted options={
      fontsize=\scriptsize,
      autogobble,
      breaklines,
      obeytabs,
      tabsize=2,
      linenos,
      numbersep=2mm,
      xleftmargin=4mm
    },
    colback=gray!5,
    left=1mm,
    right=1mm,
    top=0.5mm,
    bottom=0.5mm
  }
}


\ExplSyntaxOn
\int_new:N \g_latex_sample_counter_int
\int_set:Nn \g_latex_sample_counter_int {0}
\seq_new:N \g_latex_sample_seq

\cs_set:Nn \latex_sample_anchor_name: {
  latex@sample@\int_to_alph:n {\g_latex_sample_counter_int}
}

\cs_set_eq:NN \hypertarget:nn \hypertarget
\cs_generate_variant:Nn \hypertarget:nn {xn}

\cs_set:Npn \add_latex_sample:n #1 {
  % add tabular line
  \seq_gput_right:Nx \g_latex_sample_seq {
    \int_use:N \g_latex_sample_counter_int &
    \exp_not:N\hyperlink {\latex_sample_anchor_name:}{#1} &
    \exp_not:N\pageref* {\latex_sample_anchor_name:}
  }
  \int_use:N \g_latex_sample_counter_int
}

\AtEndDocument{
  \iow_open:Nn \g_tmpa_iow {\jobname-latex-examples.tabular}
  \iow_now:Nx \g_tmpa_iow {\c_left_brace_str}
  \iow_now:Nn \g_tmpa_iow {\setlength{\tabcolsep}{0.02\linewidth}}
  \iow_now:Nn \g_tmpa_iow {\begin{supertabular}{@{}p{0.06\linewidth}
                          p{0.8\linewidth}p{0.06\linewidth}@{}}}
  \seq_map_variable:NNn \g_latex_sample_seq \l_tmpa_tl {
    \exp_args:NNV \iow_now:Nn \g_tmpa_iow \l_tmpa_tl
    \iow_now:Nn \g_tmpa_iow {\tabularnewline}
  }
  \iow_now:Nn \g_tmpa_iow {\end{supertabular}}
  \iow_now:Nx \g_tmpa_iow {\c_right_brace_str}
  \iow_close:N \g_tmpa_iow
}

\newcommand{\listoflatexsamples}{
  \section*{List~of~Examples}
  \IfFileExists{\jobname-latex-examples.tabular}{
    \input{\jobname-latex-examples.tabular}
  }{}
}

\newtcblisting{latexsample}[1]{
  codesample,
  title={
    \int_gincr:N \g_latex_sample_counter_int
    \exp_args:Nx \label {\latex_sample_anchor_name:}
    \hypertarget:xn {\latex_sample_anchor_name:}{
    \small\textbf{Example~\add_latex_sample:n {#1}}:~#1}
  }
}

\ExplSyntaxOff

% anonymous example
\newtcblisting{latexsample*}{
  codesample
}

% listing only example
\newtcblisting{latexsample**}{
  codesample,
  listing only
}

% listing of any language
\newtcblisting{codesample}[1]{
  codesample,
  listing only,
  minted language=#1
}


\title{\liii: Programming in \LaTeX\ with Ease}
\author{Ziyue ``Alan'' Xiang}
\address{Purdue University}
\netaddress{ziyue.alan.xiang (at) gmail (dot) com}
\personalURL{https://www.alanshawn.com}
\ORCID{0000-0001-6054-5801}


\begin{document}

\begin{abstract}

\end{abstract}

\maketitle

\tableofcontents
\listoflatexsamples

\section{Introduction}

Many people view \LaTeX\ as a typesetting language and overlook
the importance of programming in document generation process. 
In reality, many large and structural documents can benefit from
a programming backend, which enhances layout standardization, symbol
coherence, editing speed and many other aspects. Despite the fact the
standard \LaTeX\ (\LaTeXe) is already Turing complete, which means
it is capable of solving any programming task, the design of numerous
programming interfaces is highly inconsistent due to the long history
of \LaTeX. This makes programming with \LaTeXe\ extremely daunting, 
even for seasoned computer programmers.

To make programming in \LaTeX\ easier, the \liii\ programming interface
is introduced, which aims to provide modern-programming-language-like
syntax and library for \LaTeX\ programmers. Unfortunately, learning
materials for this wonderful language is scarce. One of the few resource
available for new users is \emph{The \liii\ Interfaces} \cite{l3interface},
which is essentially an API documentation that is not designed for 
introductory purposes. This situation may have barred many \LaTeX\ 
users from utilizing the generic programming capabilities of \LaTeX.
Therefore, this article intends to provide an easy-to-understand 
tutorial for \LaTeX\ users with computer programming knowledge.
Hopefully, readers can improve their \LaTeX\ editing efficiency and 
document quality after understanding \liii.

This article is largely based on \emph{examples}, which demonstrate
different aspects of \liii\ programming. The minimum preamble for 
all examples is
\begin{latexsample**}
\documentclass{article}
\usepackage{tikz} % load TikZ for some TikZ examples
\usepackage{expl3} % load latex3 packages
\end{latexsample**}
\noindent Examples should be placed between \inltex|\begin{document}| 
and \inltex|\end{document}|. All examples are tested with \TeX Live 
2020 on Ubuntu 20.04.

Since the scale of \liii\ is huge, it would be infeasible to cover
all modules and functionalities in one tutorial. As a result, this
article only focuses on most frequently used components in \liii. 
The complete API documentation can be found in \emph{The \liii\ Interfaces}
\cite{l3interface}. The numbers surrounded by parentheses in some 
section titles of this tutorial indicate the corresponding 
documentation location in \emph{The \liii\ Interfaces}.

\subsection{Why \liii?}

As mentioned above, \LaTeXe\ is already Turing complete and serves as
the building block of many important packages. What are the reasons 
for switching to \liii?

\paragraph{Better expansion control} Fundamentally, \LaTeX\ works by 
doing \emph{substitution}: commands are substituted by their definition,
which is subsequently replaced by definition's definition, until 
something irreplaceable is reached (e.g. text or \TeX\ primitives). 
This process is called \emph{expansion}. The mechanism of expansion
may sound simple and straightforward. However, it usually requires
a lot of manual fine-tuning in practice.

Consider the example below. We know that the \inltex|\uppercase| macro
capitalize English letters, which renders the first output line in all 
caps. But if we store some text in \inltex|\myname| and then apply
\inltex|\uppercase| to the command, we can see that the output is
\emph{not} turned into uppercase letters.

\begin{latexsample*}
\par\uppercase{Alan Xiang}
\newcommand*{\myname}{Alan Xiang}
\par\uppercase{\myname}
\end{latexsample*}

Why would this happen? Let us dig into how \inltex|\uppercase| works. 
The \inltex|\uppercase| macro scans each token\footnotemark inside its argument 
group one by one. If an English letter is encountered, its uppercase 
form is left in the output stream. If a command is encountered, it 
will not try to apply \inltex|\uppercase| to the content of the command. 
Instead, the command itself will be placed into the output stream. 
In this case, \inltex|\myname| will be left untouched in the output, 
which is subsequently expanded to its original definition.

\footnotetext{Tokens are smallest units that \TeX\ compilers work 
with. For now, we can consider a token to be either a character 
or command. For more about \TeX\ tokens, see \cite{overleaf-token}.}

What if we also want to capitalize the content of \inltex|\myname| as 
well? To achieve this, we need to fine-tune the expansion process by 
changing the \emph{order} of expansion. That is, to expand 
\inltex|\myname| before \inltex|\uppercase|. In this way, the 
\inltex|\uppercase| command will receive the content of 
\inltex|\myname| in the form of English letters, which allows 
capitalization to function correctly.

In \LaTeX, the classic way of controlling the order of expansion is
via the \inltex|\expandafter| macro, which it is notoriously
difficult to use. According to \emph{A Tutorial on \cs{expandafter}} 
\cite{bechtolsheim1988tutorial}, to reverse the expansion of a series
of $n$ tokens, the $i$th token has to be preceded by $2^{n-i}-1$
\inltex|\expandafter|s. The exponential growth of the number of 
\inltex|\expandafter|s greatly reduces the readability of source code
and increases the chances of mistakes. 
For example, in Joseph Wright's answer to an expansion-related
question on \TeX\ StackExchange \cite{tex-se-expanding}, a total of 26 
\inltex|\expandafter|s are used to reorder the expansion of merely 4
arguments.  To avoid this annoyance, one of the key features
of \liii\ is to provide simple and reliable expansion control.

\paragraph{Standardized interface} Just like any other generic programming
languages, \LaTeX\ provides integer, floating point, string and
container variables. However, tradition \LaTeX\ interface for these types
is very messy. For example, to compare the equality of two strings, we can
use \inltex|\ifthenelse| and \inltex|\equal| from \verb|ifthen| package;
we can use \inltex|\pdfstrcmp| from \verb|pdftexcmds| package; we can also
use \inltex|\IfStrEq| from \verb|xstring| package. The fact that so many
heterogeneous packages provide similar functionalities induces redundancy
and potential compatibility issues. Therefore, \liii\ is to provide
a set of unified and standardized interfaces for all possible \LaTeX\ 
variable types.

\paragraph{Modernized experience} \TeX\ was first designed in the late 1970s, when
computer hardware and programming languages were prototypes compared to
their contemporary counterparts. As a result, \TeX\ and \LaTeX\ contain
quirky usages that may seem odd for programmers today. For example, to
multiply a counter variable by 3, one writes 
\inltex|\multiply|\inltex|\counter| \inltex|by| \inltex|3|; to invoke the \verb|date| command via
the terminal, one writes \inltex|\immediate\write18{date}|. 
It can be seen that these syntaxes are either outdated or perplexing. 
In a fairly popular language nowadays 
(e.g. Python), these two tasks can be done by \inlpy|counter*=3|
and \inlpy|os.system('date')|, whose code possesses superior
simplicity and interpretability. \liii\ attempts to modernize 
the \LaTeX\ language by adapting to modern-language-like syntaxes
and introducing a naming system that makes \LaTeX\ code more
readable.

\section{\liii\ Naming Conventions (I-1)}

In Python or C++, if we see \inlpy|a(b);|, we can 
tell \inlpy|a| is a function and \inlpy|b| is its 
argument. However, in \LaTeX, if we see \inltex|\a\b|, 
there are be two possibilities:
\begin{itemize}
\item \inltex|\a| is a function and \inltex|\b| is its argument
\item Both \inltex|\a| and \inltex|\b| are variables
\end{itemize}
The syntactic design of \LaTeX\ makes it difficult
to distinguish between functions and variables, for
each control sequence can either be a function that 
receives arguments or a variable that absorbs nothing.
It can lead to confusion when one is trying to understand
others' source code. Therefore, \liii\ introduces a set
of naming rules that encode important information into
the name of control sequences as a way to improve 
readability.

Before discussing \liii\ naming conventions, let us take a diversion 
to look at the low-level design of \LaTeX\ and find out 
how we can use non-English characters in command names.

\subsection{Category Code \& Command Name}

When the \LaTeX\ compiler reads a source file, it will read
and process each character one by one. For each character in 
the file, in addition to its character code, \LaTeX\ compiler 
will also assign a \emph{category code} based on current
category code table. The default \LaTeX\ category code table
is shown in Table \ref{table:catcode-table}.

\begin{center}
\tabletail{\hline}
\tablehead{
\hline
\makecell{Category\\Code} & Description & 
\makecell{Character(s)}\\ \hline
}
\bottomcaption{Default \LaTeX\ category code table \cite{overleaf-catcode}. 
Characters surround by single quotes indicate their C-style
representation.}
\label{table:catcode-table}
\begin{supertabular}{|c|>{\centering}m{0.4\linewidth}|c|}
0 & Escape character: tells \LaTeX\ 
to start looking for a command & \verb|\| \\  \hline
1 & Start of group & \verb|{|\\ \hline
2 & End of group & \verb|}|\\ \hline
3 & Toggle math mode & \verb|$|\\ \hline
4 & Alignment tab & \verb|&|\\ \hline
5 & End of line & \verb|'\r'|\\ \hline
6 & Macro parameter & \verb|#| \\ \hline
7 & Superscript & \verb|^| \\ \hline
8 & Subscript & \verb|_| \\ \hline
9 & Ignored character & \verb|'\0'| \\ \hline
10 & Spacer & \verb|'\32'|, \verb|'\t'|\\ \hline
11 & Letter & \verb|A|--\verb|Z|, \verb|a|--\verb|z|, \ldots\\ \hline
12 & Other & \verb|0|--\verb|9|, \verb|+|, \verb|@|\ldots \\ \hline
13 & Active character: used for single character commands & 
\verb|~|\ldots\\ \hline
14 & Comment character: ignore everything that follows until 
end of line & \verb|%|\\ \hline
15 & Invalid character: not allowed in \verb|.tex| files & 
\verb|'\127'|\ldots \\ \hline
\end{supertabular}
\end{center}

\LaTeX\ reacts to each character according to its category code
instead of character code. If we change the category code associated
with a character, we can completely change the \emph{meaning} of that
character. For example, if we assign category code 7 to \verb|_| and
category code 8 to \verb|^|, we can use \verb|_| to denote superscript
and \verb|^| to denote subscript.

\begin{latexsample}{Doing 123}
\ExplSyntaxOn
\tl_set:Nn \l_tmpa_tl {A}
\group_begin:
\tl_set:Nn \l_tmpa_tl {B}
\par value~inside~group:~\tl_use:N \l_tmpa_tl
\group_end:
\par value~outside~group:~\tl_use:N \l_tmpa_tl

\tl_set:Nn \l_tmpb_tl {A}
\group_begin:
\tl_gset:Nn \l_tmpb_tl {B}
\par value~inside~group:~\tl_use:N \l_tmpb_tl
\group_end:
\par value~outside~group:~\tl_use:N \l_tmpb_tl
\ExplSyntaxOff
\end{latexsample}



\bibliographystyle{tugboat}
\bibliography{main.bib}

\makesignature

\end{document}