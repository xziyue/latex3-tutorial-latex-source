\section{Understanding \LTT{} Documentation}

Currently, most information about \LTT{} is in \cite{l3interface}. 
It is essentially technical documentation on \LTT{} functions, scratch variables, and constants. 
We briefly describe how these items are presented in \cite{l3interface}.
\begin{itemize}
\item \LTT{} functions: the majority of the contents in \cite{l3interface} are about \LTT{} functions.
In \cite{l3interface}, each function is documented using a block similar to the one shown in  \cref{fig:l3-doc-func-example}.
The colored background is added in the figure to help understand the block.
The green part in \cref{fig:l3-doc-func-example} indicates the basic forms and the variants of the function. 
The first and third line in the green part are the basic forms for local and global assignment, respectively.
The second line shows the function variants for local assignment; the fourth line shows the function variants for global assignment.
%TODO: insert reference and delete Section
Function variants are used to control the expansion of macros. 
More details about them will be introduced in \refplaceholder.
The blue part in \cref{fig:l3-doc-func-example} shows the basic usage of the function, where the meaning of each function argument is shown.
The red part in \cref{fig:l3-doc-func-example} shows the detailed description of the function.

\item Scratch variables: variables in \LTT{} need to be defined before use. 
\LTT{} has predefined a set of empty variables for convenience, which are known as scratch variables.
In \LTT{}, each module will usually define several scratch variables.
They are documented in specific sections, similar to the one demonstrated in \cref{fig:l3-scratch-var-example}.
The left hand side of \cref{fig:l3-scratch-var-example} shows the predefined scratch variables; the right hand side shows the description of the variables.
It is recommended to not use scratch variables (especially in large projects or generic packages), which can reduce the chances of variable collision.

\item Constants: \LTT{} defines many constants such as the value of $\pi$, the value of $e$, special characters, etc. 
They are documented in specific sections, similar to the one demonstrated in \cref{fig:l3-constant-example}.
The left hand side of \cref{fig:l3-constant-example} shows the command name of the constant; the right hand side shows the description of the constant.

\end{itemize}

\begin{figure*}[tb]
\centering
\mdfdefinestyle{plainmdbox}{topline=false,leftline=false,rightline=false,bottomline=false}
\begin{minipage}[t]{0.52\linewidth}
\begin{mdframed}[backgroundcolor=green!30, style=plainmdbox]
\begin{tabular}{l}
\toprule
\inlpl|\tl_set:Nn|\\
{\color{gray}\verb|\tl_set|}\inlpl!:(NV|Nv|No|Nf|Nx|cn|cV|cv|co|cf|cx)! \\
\inlpl|\tl_gset:Nn| \\
{\color{gray}\verb|\tl_gset|}\inlpl!:(NV|Nv|No|Nf|Nx|cn|cV|cv|co|cf|cx)! \\ \bottomrule
\end{tabular}
\end{mdframed}
\end{minipage}
\begin{minipage}{0.4\linewidth}
\vspace*{3em}
\begin{mdframed}[backgroundcolor=blue!30, style=plainmdbox]
\inlpl|\tl_set:Nn| {\ttfamily<\textsl{tl var}> \{<\textsl{tokens}>\}}
\end{mdframed}
\end{minipage}\\
\begin{minipage}[t]{0.15\linewidth}
\phantom{}
\end{minipage}
\begin{minipage}{0.82\linewidth}
\begin{mdframed}[backgroundcolor=red!30, style=plainmdbox]
Sets <\textit{tl var}> to contain <\textit{tokens}>, removing any previous content from the variable.
\end{mdframed}
\end{minipage}
\caption{An example of function documentation excerpted from \LTT{} documentation. 
Colored background is added in this figure to help understand the documentation.}
\label{fig:l3-doc-func-example}
\end{figure*}

\begin{figure*}[tb]
\centering
\begin{minipage}[t]{0.15\linewidth}
\begin{tabular}{l}
\toprule
\inlpl|\l_tmpa_tl|\\
\inlpl|\l_tmpb_tl| \\ \bottomrule
\end{tabular}
\end{minipage}
\begin{minipage}[t]{0.83\linewidth}
\vspace*{-1em}
Scratch token lists for local assignment.
These are never used by the kernel code, and so are safe for use with any \liii--defined function.
However, they may be overwritten by other non-kernel code and so should only be used for short-term storage.
\end{minipage}
\caption{An example of scratch variables excerpted from \LTT{} documentation.}
\label{fig:l3-scratch-var-example}
\end{figure*}

\begin{figure*}[tb]
\centering
\begin{minipage}[t]{0.15\linewidth}
\begin{tabular}{l}
\toprule
\inlpl|\c_pi_fp|\\ \bottomrule
\end{tabular}
\end{minipage}
\begin{minipage}[t]{0.83\linewidth}
The value of $\pi$. This can be input directly in a floating point expression as \texttt{pi}.
\end{minipage}
\caption{An example of constant variables excerpted from \LTT{} documentation.}
\label{fig:l3-constant-example}
\end{figure*}
